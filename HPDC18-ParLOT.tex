\documentclass[conference]{IEEEtran}
\IEEEoverridecommandlockouts
% The preceding line is only needed to identify funding in the first footnote. If that is unneeded, please comment it out.
\usepackage{xspace}
\usepackage{cite}
\usepackage{amsmath,amssymb,amsfonts}
\usepackage{algorithmic}
\usepackage{graphicx}
\usepackage{color}
\usepackage{textcomp}
\usepackage[normalem]{ulem}
\usepackage{soul}
\usepackage{multirow}
\usepackage[table,xcdraw]{xcolor}
%--
\newcommand{\ignore}[1]{ } % ignore blocks of text
\newcommand{\maketiny}[1]{{\tiny G: #1}} % ignore blocks of text
\newcommand{\ganesh}[1]{\todo[inline,size=\small, color=purple!40]{G: #1}}

\newcommand{\parlot}{\textsc{ParLoT}\xspace}
\newcommand{\pin}{\textsc{PIN}\xspace}
\newcommand{\callgrind}{\textsc{Callgrind}\xspace}

%--\newcommand{\circleRtool}{circleRtool\circledR\xspace}

%%% Local Variables:
%%% mode: latex
%%% eval: (flyspell-mode 1)
%%% TeX-master: "root.tex"
%%% End:

%--
\def\BibTeX{{\rm B\kern-.05em{\sc i\kern-.025em b}\kern-.08em
    T\kern-.1667em\lower.7ex\hbox{E}\kern-.125emX}}
\begin{document}

\title{\parlot: Low-Overhead Tracing of Large-Scale Parallel Programs\\
\thanks{This work was supported in part by NSF grant 1438963.}
}

\author{
  \IEEEauthorblockN{Saeed Taheri}
        \IEEEauthorblockA{School of Computing\\
                University of Utah\\
                Salt Lake City, Utah, U.S.A.\\
                Email: staheri@cs.utah.edu}
        \and
	\IEEEauthorblockN{Sindhu Devale}
	\IEEEauthorblockA{Department of Computer Science\\
		Texas State University\\
		San Marcos, Texas, U.S.A.\\
		Email: sindhu.devale@gmail.com}
	\and
	\IEEEauthorblockN{Ganesh Gopalakrishnan}
	\IEEEauthorblockA{School of Computing\\
		University of Utah\\
		Salt Lake City, Utah, U.S.A.\\
		Email: ganesh@cs.utah.edu}
	\and
	\IEEEauthorblockN{Martin Burtscher}
	\IEEEauthorblockA{Department of Computer Science\\
		Texas State University\\
		San Marcos, Texas, U.S.A.\\
		Email: burtscher@cs.txstate.edu}
}

\maketitle

\begin{abstract}
\label{abs}
The need for efficient tracing to gain better insight from
HPC application executions is growing. 
%
An efficient and easy-to-use tracing method that overcome the trade-off
between maximum information with minimum added overhead
sounds essential for today's HPC application developers, who might
not have enough expertise in parallel programming.
% 
Unfortunately, available tools either do not produce the kinds
of traces that can help with program understanding and debugging,
or incur huge overheads when
the traces are brought out of the cores.
%
%
In this paper, we present \parlot, a tool that provides several key
features: (1)~It dynamically instruments the binary without need of source-code modification or recompilation
%
(2)~It deploys a set of highly efficient trace compression methods that reduces the trace volume gathered at runtime dramatically, thus making tracing cheaper.
%
(3)~It supports program execution analysis via tracing by
including caller/callee
relations, call frequencies, and a linear trace of entire executions
at the granularity the user opts for.
%
This paper establishes that comparable capabilities are 
unavailable by evaluating the best alternative tracing options
on runs up to 1,024 cores on the NAS parallel benchmarks on the
Comet supercomputer.
%
Our experiments show that \parlot can produce whole-program function 
call traces with average of 56 kB/s required bandwidth while 
slowing down applications by 2.7x on average. 

\end{abstract}


\begin{IEEEkeywords}
tracing, HPC, compression;
\end{IEEEkeywords}

\extrafloats{100}

\section{Introduction}
\label{sec:intro}
When the next version of an HPC software system is created, logical errors often
get introduced.
%
To maintain productivity, designers need effective and efficient methods to locate
these errors.
%
Given the increasing use of hybrid (MPI + X) codes and library functions, errors may
be introduced through a usage contract violation at any one of these interfaces.
%
Therefore, tools that record activities at multiple APIs are necessary.
%
Designs find most of these bugs manually, and the efficacy of a debugging tool is
often measured by how well it can highlight the salient differences between the
executions of two versions of software.
%
Given the huge number of things that could be different -- individual iterative
patterns of function calls, groups of functions calls, or even specific instruction
types (e.g., non-vectorized versus vectorized floating-point dot vector loops) -- designers
cannot often afford to rerun the application multiple times to collect each facet
of behavior separately.
%
These issues are well summarized in many recent studies cite-them.

TODO: Cite the DOE debugging report.


TODO: Talk about the challenges of debugging in the context of heterogeneous programs. (NSF proposal material.)

TODO: Challenges in hybrid debugging and ``always on'' call for
collecting traces efficiently.

- binary tracing helps here

- detect at any level of abstraction, as bugs may be present at different levels, APIs

 * filter

 * attribute creation

TODO: Express that debugging is a matter of dissimilarity finding.

  * Dissimilarities are what bugs are

  * Dissimilarity must be found in richer spaces

We have created 2 such spaces

* Large space - need to summarize across processes, then threads

    CL - Jaccard - heatmap - metrics

  * TODO: most programs spend most of their time in loops.

    Loop structure - loop diff

**   TODO: Highlights of results obtained as a result of the above thinking should be here. THis typically
comes before ROADMAP of paper.

** restate key contributions in a few bullets

ONE KEY CONTRIB is that we have a powerful combination of ideas to locate bugs

1. JS based location of i,j pair that matters

2. NLR-based identification of WHY it matters (and is a bug)

EVAL method:
initial recommendation turns into NLR-observable bug and confusion matrix scores how many TP,FP,TN,FN


TODO: Brief roadmap of paper

- Sec Background.tex here. sec2 talks about ... . sec3 ...

- Sec major related work will have a detailed discussion of related work

- this probably can come after your Evaluation section...
 

% \begin{itemize}
% \item HPC Bugs are expensive
% \item HPC Tracing is expensive
% \item ParLOT provides infrastructure for whole-program function call return tracing
% \item Decompressing traces produce large amount of data
% \item Smart analysis of traces, convert large amount of data from thousands of traces (processes, threads) into a valuable information which are
% 	\begin{itemize}
% 	\item Easy to analyze
% 	\item hide the complexity of HPC
% 	\item Reflect dynamic behavior of the program
% 	\item see if it matches the user/developer expectations
% 	\end{itemize}
% \item The basic idea is to reduce the search space and reflect abnormal behavior by:
% 	\begin{itemize}
% 	\item Classifying similar behavior traces into equivalent classes
% 	\item Comparing representatives from equivalent classes and find the point of divergence (i.e., diffing)
% 	\end{itemize}
% \item Here are the contributions:
% 	\begin{itemize}
% 	\item Testing framework (diffTrace) from fault injection to large-scale applications to diffing pairs of candidate trace files causing the fault. (clTrace and fpTrace)
% 	\item MPI fault injection
% 	\item ParLOT trace Collection
% 	\item Trace Preprocessing (decompression, loop detection, attribute extraction)
% 	\item Concept Lattice, Jaccard Similarity Matrix, Trace classification, outlier detection
% 	\item Diffing
% 	\item Floating Point Debugging
% 	\end{itemize}
% \end{itemize}
%




\section{Background}
\label{sec:background}
Like any program analysis tool, a mechanism is required to collect data about the program behavior.

\begin{itemize}
\item \hl{Static vs. Dynamic}
\item \hl{Sampling and hardware counters like TAU and Score-p vs Whole program analysis}
\item \hl{pros and cons of each, a small fraction of related work on tracing}
\end{itemize}

We have used ParLOT \cite{parlot} to collect our data.
%
ParLOT is an efficient tracing tool that captures all function calls and returns at different levels via dynamic binary instrumentation \cite{pin}.
%
ParLOT incrementally compresses the captured function calls and returns on-the-fly, resulting in highly compressed trace files that occupy only a few kilobytes on the disk and leave the majority of bandwidth for the network and system.
%
Upon termination of the application execution, either a ``successful’’ termination or ``failure’’ of the application due to crash, deadlock (T/O) or corrupted result, a Compressed ParLOT Trace (CPT) for each executing thread would be generated. Each $CPT_i$ then be \textit{pre-processed} (e.g., decompressed, filtered, etc. more in section \ref{subsec:nlr}) and form  $PT_i$ where $i$ is the index of process/thread.
%
Our strategy is to perform a post mortem analysis on PTs to study different aspects of application dynamic behavior.
%
Our main focus is the comparison between a successful run and failed run.
%
Such an approach has proven to be useful in works such as delta debugging \cite{choi-concurentDelta} where an application used to work fine yesterday but failed today due to various factors such as a change in the source code, an API library update, use of a different compiler or porting to a new system.

\subsection{Odd/Even Sort Example}

For a better explanation of our idea, we explain our approaches on a simple parallel (MPI) implementation of odd/even sort \hl{[cite the git url]}.

Odd/Even sort is a variant of the bubble-sort operates in two alternate phases: \textit{Phase-even} where even processes exchange (compare and swap) values with right neighbors and \textit{Phase-odd} where odd processes exchange values with right neighbors. Figure \ref{fig.oddEven} shows the simplified MPI implementation of the odd/even sort algorithm.


The for loop in line 4 of \texttt{oddEvenSort()} iterates over phases of the algorithm and based on the phase, the appropriate partner for each rank is getting discovered by the function \texttt{findPtr()} (line 6). The odd/even ranks then exchange their chunks of data (lines 9-13) and a set of sort, merge and copy operations would be performed on received data by each rank (which are replaced by \texttt{...} in line 15 for simplicity).

\begin{figure}[]
\centering
\caption{Simplified MPI implementation of Odd/Even Sort}
\includegraphics[width=0.45\textwidth]{figs/oddEven.png}
\label{fig.oddEven}
\end{figure}


Execution of odd/even sort application with four processes (\texttt{mpirun -np 4}) while ParLOT trace collection is enabled on top of the application, would result in 4 PTs (table \ref{tab:oddEvenPT}). This execution terminates successfully with expected results and the set of generated PTs clearly reflects the expected behavior (control flow) of odd and even processes.
%
\\
Now let's swap the the order of MPI\_Recv and MPI\_Send in lines 11 and 12 (figure \ref{fig.oddEvenDL}). According to MPI Standard  \hl{[cite MPI-forum or openMPI url]}, MPI\_Send is a \textit{blocking send} used the \textit{standard} communication mode. In this mode,  MPI may buffer outgoing messages and the send call may complete before a matching receive is invoked. On the other hand, buffer space may be unavailable, or MPI may choose not to buffer outgoing messages, for performance reasons. In this case, the send call will not complete until a matching receive has been posted, and the data has been moved to the receiver. This shows that, based on the MPI implementation, the \texttt{oddEvenSort\_DL()} might end up causing a deadlock.

\input{tabs/oddEvenPT.tex}

\begin{figure}[]
\centering
\caption{A line change in oddEvenSort (left) that might cause a deadlock in oddEvenSort\_DL (right)}
\includegraphics[width=0.45\textwidth]{figs/oddEvenDL.png}
\label{fig.oddEvenDL}
\end{figure}


HPC applications often execute on supercomputers with multiple levels of parallelism from distributed systems (MPI) to shared memory (OpenMP) and accelerators (GPU). Finding flaws like the one in figure \ref{fig.oddEvenDL} in such an environment is the problem of \textit{finding the needle in the haystack}.
%
\begin{itemize}
\item Trade-off between collecting sufficient information and adding reasonable overhead to the native execution is solved with ParLOT
\item However, generated PTs by ParLOT are difficult to analyze (thousands of long sequences of function calls and returns)
\item Solution:
	\begin{itemize}
	\item intra-PT compression (decompress, filter and NLR: section \ref{subsec:nlr})
	\item inter-PT compression (Concept Lattices: section \ref{subsec:fca})
	\item Locating suspicious PTs that might suffer from a flaw: section \ref{subsec:ranking} and delta comparison (diffNLR section: \ref{subsec:diffnlr})
	\end{itemize}
\end{itemize}


\clearpage

% NLRs
\subsection{Pre-Processing CPTs}
\label{subsec:nlr}

\begin{itemize}
	\item Decompress and Filters
	\begin{itemize}
		\item ParLOT leaves Compressed Traces (CPTs)
		\item CPTs need to be decompressed and decoded to PTs 
		\item Each PT contains a full sequence of ordered function calls. Not all of the functions are in our interest - Hierarchy of filters: Table \ref{tab:filters}.  
	\end{itemize}
	\item Loops in source codes would be reflected in PTs as sequences of repetitive patterns.
	\item HPC applications and resources are of great interest to scientists and engineers for simulating \textit{iterative} kernels. Computer simulation of fluid dynamics, partial differential equations, the Gauss-Seidel method, and finite element methods in form of stencil codes, all include a main outer loop that iterates over some elements (i.e., timesteps) and updates the elements.
	\item Mining loops advantages:
	\begin{itemize}
		\item Lossless compression (intra-PT), making long PTs easier to analyze and read
		\item broken loop structures due to a fault would be revealed
		\item Others have done the same (STAT, AutomaDeD, PRODOMETER) 
	\end{itemize}
	\item We have adopted ideas from Kobayashi \cite{kobayashi-84} (definition of loops) and \cite{Ketterlin-nlr} (NLR algorithm)
	\item We have slightly changed the NLR algorithm to make it suitable for PT contents.
	 
	\item explanations on loop structures, loop bodies and loop counts, example \ref{fig.NLRexample}
	\item NLR algorithm and complexity
\end{itemize}

\input{tabs/filters.tex}


\begin{figure}[]
\caption{Sample NLR}
\includegraphics[width=0.45\textwidth]{figs/NLRexample.png}
\label{fig.NLRexample}
\end{figure}

\clearpage

\subsection{Equivalencing behavior via FCA}
\label{subsec:fca}

Thanks to ParLoT compression mechanism, we are able to efficiently (w.r.t. time and space) collect whole-program function call and return traces (PTs). However, post-mortem analysis of the PTs from thousands of threads requires decompression of traces, and consequently, analysis of large amount of data. Before jumping into \textit{the huge haystack} of PTs to find \textit{the tiny needle} (bug, bug manifestation or root cause of the failure), a middle ground data manipulation is required to simplify and organize the haystack. 

Reducing the search space from thousands of PTs to just a few groups of equivalent PTs (i.e., inter-PT compression) not only requires a similarity measure based on a call matrix but also a scheme that is efficient even for large process counts.
%
Since a pair-wise comparison of all processes is highly inefficient, we use \textit{concept lattices} that stem from \textit{formal concept analysis} (FCA) \cite{clbook} to store and compute groups of similar PTs.
%
FCA can efficiently split the large haystack into a few hay(semi)stacks with ``conceptually'' similar hays in each. This way conceptually isolated PTs (i.e., outliers) which are the potential bug manifestation or root cause would be detected. If no outlier detected, we only have a few distinct group of PTs to dig in, instead of thousands of large traces. With a wider perspective, here are other benefits of FCA for HPC debugging:
\begin{itemize}
\item FCA is scalable and efficient. It can be built incrementally and different kind of information such as full Jaccard Similarity Matrix (JSM) can be generated in linear time due to CL properties.
\item Clustering is only one advantage of creating concept lattices from ParLoT traces. CLs can integrate all traces from an execution to a single entity as signature/model of good or bad execution for further analysis (e.g., prediction) 
\item Due to the \textit{partial order} of nodes within CLs, valuable information can be retrieved from CLs like Happens-Before relation (Vijay Garg’s book explains all applications of FCA in computer science applications)\cite{latticeForDistConst} and machine learning and data mining \cite{Ignatov17})
\end{itemize}

A concept lattice is based on a \textit{formal context} \cite{clbook}, which is a triple $(O, A, I)$, where $O$ is a set of \textbf{objects}, $A$ a set of \textbf{attributes}, and $I \subseteq O \times A$ an incidence relation. The incidence relation associates each object with a set of attributes (e.g., table \ref{tab:sampleContext}).
%
Using FCA for clustering giving us the capability of clustering trace objects based on the ``concept'' of each trace object. We can characterize the ``concept'' (i.e., what we want to understand from the collected traces) by extracting meaningful ``attributes'' from traces. 
%
However, since we are only interested in grouping similar PTs in this work, we only take advantage of similarity measures \cite{Alqadah2011} of concept lattices and leave other properties for future work.
%
Due to typical HPC application topologies such as SPMD, master/worker and odd/even where multiple processes/threads behave similarly, our experiments show that large numbers of PTs can be reduced to just a few groups.
%

\subsubsection{Concept Lattice Construction}
\begin{itemize}
\item Batch vs. Incremental \cite{clconst}
\item Complexity: $O(2^{2K}||E||)$ where $K$ is an upper bound for number of attributes (e.g., distinct function calls in the whole execution) and $||E||$ is the number of objects (e.g., number of PTs).
\end{itemize}

\subsubsection{FCA example}

\begin{itemize}
	\item Construct the context table from example in figure \ref{fig.oddEven}
	\item Construct the Actual Concept Lattice from example in figure \ref{fig.oddEven}
\end{itemize}



\begin{table}[]
\label{tab:sampleContext}
\caption{Context}
\scalebox{0.6}{
\begin{tabular}{l|cccccc}
 & \multicolumn{1}{l}{MPI\_Init()} & \multicolumn{1}{l}{MPI\_Comm\_Size()} & \multicolumn{1}{l}{MPI\_Comm\_Rank()} & \multicolumn{1}{l}{MPI\_Send()} & \multicolumn{1}{l}{MPI\_Recv()} & \multicolumn{1}{l}{MPI\_Finalize()} \\ \hline
Rank 0 & $\times$ & $\times$ & $\times$ &  & $\times$ & $\times$ \\
Rank 1 & $\times$ & $\times$ & $\times$ & $\times$ &  & $\times$ \\
Rank 2 & $\times$ & $\times$ & $\times$ & $\times$ &  & $\times$ \\
Rank 3 & $\times$ & $\times$ & $\times$ & $\times$ &  & $\times$
\end{tabular}}
\end{table}


\begin{figure}[t]
\centering
\scalebox{0.5}{
\includegraphics[width=3.4in]{figs/{sample}.pdf}}
\caption{Sample Concept Lattice from Obj-Atr Context in table\ref{tab:sampleContext}}
\label{fig:sampleCL}
\end{figure}

\begin{figure}[t]
\centering
\scalebox{0.5}{
\includegraphics[width=3.4in]{figs/{sample-reduced}.pdf}}
\caption{Concept Lattice with reduced labels}
\label{fig:sampleCL}
\end{figure}





\subsubsection{Jaccard Similarity Scores}

\begin{itemize}
\item Some background about Jaccard Similarity Score
\item How to obtain full pair-wise Jaccard Similarity Matrix (JSM) from a concept lattice (e.g., LCA approach)
\end{itemize}

\begin{figure}[]
\centering
\scalebox{0.8}{
\includegraphics[width=3.4in]{figs/{fancy1}.pdf}}
\caption{Pair-wise Jaccard Similarity Matrix (JSM) of MPI processes in Sample code}
\label{fig:jsm}
\end{figure}


%\subsection{Ranking Suspicious PTs}
%\label{subsec:ranking}
%\begin{itemize}
%\item Filters
%\item Attributes
%\item 
%\end{itemize}



\subsection{diffNLR: Reflecting differences}
\label{subsec:diffnlr}
\begin{itemize}
\item Inspired by \texttt{diff} original algorithm\cite{diff-myers} that has bin used in Git and GNU Diff, we visualize the differences of a pair of PT as shown in fig \ref{fig:sampleDiffNLR}.
\item This visualization reflects of the differences of \textbf{occurrences} of PT elements and their \textbf{orders}.
\item In section \ref{sec:experimental} we show how this visualization can help us locating the points of divergence in PTs, and potential bug manifestation and root cause.
\end{itemize}

\begin{figure}[]
\centering
\scalebox{0.5}{
\includegraphics[width=3.4in]{figs/{sampleDiffNLR}.png}}
\caption{Sample diffNLR}
\label{fig:sampleDiffNLR}
\end{figure}





\section{Related Tools}
\label{sec:reltools}
Instrumenting, profiling and tracing large-scale applications have become more popular to researchers and companies \cite{ddt} due to high demand of HPC users. 
Dyninst\cite{dyninst} is a dynamic instrumentation API which gives developers the ability to measure the performance \cite{openss}\cite{tau} and develop correctness debuggers \cite{stat}. It instruments the binary without any need of recompilation and also gives the developer the ability to attach instrumentation to a running process. VampirTrace\cite{vampirt} also uses Dyninst API to provide a library for collecting logs from program execution. 

The idea of analyzing execution traces for debugging purposes have been used in STAT\cite{stat} where it groups the processes with similar function-call stack and trying to find abnormal behavior like divergence in the function call-graph by delta debugging. The idea of tracing for debugging purposes have been used in other tools but first of all the overhead they add to the target application is high and are not straight-forward for HPC users who might not be a developer. They either need static instrumentation by inserting code-snippets and macros, or/and recompilation of the source code. 
Valgrind\cite{valgrind} is shadow value DBI framework (explained in the background section) that maps and records every register and memory value. It gives developers the capability of instrumenting system calls and instructions. Many error detectors such as \textit{Memcheck} have been built on top of Valgrind. \callgrind \cite{callgrind} is a profiling tool  on Valgrind platform that records the call history among functions in a program's run as a call-graph by measuring the number of instructions executed and their relationship to source lines. 
Intermediate generated traces by \callgrind are some numbered files contain pure ascii text. \callgrind enumerate the name of files and function calls and also, stores those numerical values as relative to previous numbers. These are the only data compression options available on \callgrind which is enabled by default. Each \callgrind trace file contains a sequence of function names (or their code) and a few other numbers for each function showing the that function relationships with other functions (caller-callee). There is a tool \textit{callgrind\_ annotate} which displays different reports from the generated traces. From the generated traces by \callgrind in my experiments, the richest report that \textit{callgrind\_ annotate} can produce is the tree of function calls with caller-callee relationship and cost of each function. Cost of each function is the number of Instruction Read which is collected during tracing by reading hardware counters. Cache simulation and branch prediction information also can be enabled to be collected and then \textit{callgrind\_ annotate} can produce different reports for cache and branch prediction. By default, cache simulation and branch prediction (which are originally from another tool Cachegrind) are disabled by \callgrind.

%and  \cite{ipm} \cite{tau} \cite{scorep} \cite{vtune}

The idea of compressing large-scale  traces have been used in \cite{eventflowgraph} for compressing performance traces and in ScalaTrace\cite{scalatrace} where it uses reptetive nature of timestep simulation in parallel scientific applications to compress traces\cite{freitag}. Only small fraction of compression is happening on-fly and the focus is on reducing inter-node communication. 


\section{Implementation}
\label{sec:impl}
\input{implementation.tex}

\section{Evaluation Methodology}
\label{sec:evalmeth}
\subsection{Experimental Setup}
We executed all of our experiments on Comet \cite{comet} supercomputer at San Diego Supercomputer Center . Comet has 1944 total number of computing nodes and each node has a Dual-Socket Intel Xeon E5-2680v3 processors with total number of 24 cores (14 on each socket) and 128 GB of memory with NSF and Lustre filesystem.\\
We evaluated our tool on NAS Parallel Benchmarks (NPB) \cite{nas}. NPB has a variety of MPI-based applications, like \textit{Conjugate Gradient} (cg) with irregular memory accesss and communication and \textit{Lower-Upper} (lu) a Gauss-Seidel solver. All NPB applications have been compiled with MVAPICH2.2.1 and -g and -O1 optimization flag.\\
We evaluated \parlot with \pin version 3.5 and \callgrind version 3.13.

\subsection{Evaluation Parameters and Metrics}
In order to evaluate \parlot and make a fair comparison with similar tools, we have to have tune our experimental configurations identically and define some metrics to measure and evaluate performance and required resources.
\subsubsection{Parameters}
\begin{itemize}
\item \textbf{\# of Nodes}: We have been running all of the experiments on 1, 4, 16 and 64 number of nodes on Comet, with using 16 cores on each node (up to 1024 cores) to evaluate the behavior of \parlot on wide range of scales.
\item \textbf{Application Input}: NPB applications can be executed on different size of inputs. We have used class size \textit{B}(small-medium) and \textit{C}(medium-large)
\end{itemize}

\subsubsection{Metrics}
In order to evaluate and make a fair comparison between similar tools, first we need to decide what specific aspects of any tool are we interested in. As discussed in earlier sections, in general, an HPC debugger that "adds less overhead while collecting more information" is a "better" debugger. Thus, we define below three metrics as the basis of our comparisons and evaluations
\begin{itemize}
\item \textbf{Tracing Overhead} is simply saying how much longer the target application takes to execute when you run a tracing tool on top of it. Average tracing overhead of different tools shows how time-efficient they are. Numeric value of tracing overhead is the median runtime out of 3 identical execution of each tool on top of each target application relative to the median runtime of corresponding application with no tracing tool on top of it (native run). 
\item  \textbf{Required Bandwidth per core (kB/s)}. Tracing tools generates traces and consumes some of the system and core bandwidth. It is measured by $ReqBW_x = TraceSize_x (KB) / (\# of cores)_x / Runtime_x (S)$
\item \textbf{Compression Ratio (CR)} is the ratio of size of decompressed buffer from compressed traces over size of intermediate compressed traces.
\end{itemize}



\subsubsection{\parlot side tools / variations of \parlot}


\begin{itemize}
\item \textbf{\parlotm} collects function call traces only from the main image of application. 
\item \textbf{\parlota} collects function call traces from the all of the images including library function calls.
\item \textbf{Pin-init} is a version of \parlot where the tracing process completely shut off/disabled. Purpose of \textit{Pin-init} is to see how much of the actual tracing overhead added to native run is caused by initializing \pin and binary instrumentation (is this correct??).
\item \textbf{\parlotnc} is basically \parlot with No Compression. It stores captured data during execution as is into disk. Purpose of this side tool is to show the impact of compression on overall performance of \parlot.
\end{itemize}

\subsubsection{\callgrind}
\begin{itemize}
\item \textbf{What is \callgrind and What does it produce?} \callgrind is built on top of Valgrind platform that records the call history among functions in a program's run as a call-graph by measuring the number of instructions executed and their relationship to source lines. 
Intermediate generated traces by \callgrind are some numbered files contain pure ascii text. \callgrind enumerate the name of files and function calls and also, stores those numerical values as relative to previous numbers. These are the only data compression options available on \callgrind which is enabled by default. Each \callgrind trace file contains a sequence of function names (or their code) and a few other numbers for each function showing the that function relationships with other functions (caller-callee). Using \textit{callgrind\_ annotate} tool which displays different reports from Callgrind generated traces, the richest report that \textit{callgrind\_ annotate} can produce is the tree of function calls with caller-callee relationship and cost of each function. Cost of each function is the number of Instruction Read which is collected during tracing by reading hardware counters. Cache simulation and branch prediction information also can be enabled to be collected and then \textit{callgrind\_ annotate} can produce different reports for cache and branch prediction. By default, cache simulation and branch prediction (which are originally from another tool Cachegrind) are disabled by \callgrind.
\item \textbf{Why did we pick \callgrind to compare with}. \parlot uses a DBI to instrument and collect function-call traces to be used for debugging purposes. Surprisingly, no tool out there generates traces for debugging. \callgrind is the closest tool that we found that uses Valgrind DBI to instrument and collect function call graphs with more limited (and different) information in their traces which is more useful for beginners to see what portion of the execution time has been spent on each function. Quality (information) of collected data using \callgrind can somehow be compared with \parlot (main). However, scope of function calls that \callgrind collects is at line-of-source-code level.

\end{itemize}


\section{Results}
\label{sec:results}
%%%%%%%%%%%%%%%%%%%%%%%%%%%%%%%%%%%%%
% Slowdown vs Callgrind
%%%%%%%%%%%%%%%%%%%%%%%%%%%%%%%%%%%%%




\input{tabs.comet/comet_sd_pMpAcg_BC_int_p3.5.tex}


\input{tabs.comet/comet_sd_pMpAcg_BC_itn_p3.5.tex}

\subsection{ParLOT vs. Callgrind}

\begin{itemize}
\item \textbf{Slowdown}: Table \ref{comet_sd_pMpAcg_BC_int_p3.5} shows the slowdown of \parlot (pinMain and pinAll) and \callgrind. It shows slowdowns of same configuration (number of nodes/cores and size of input) next to each other so that we can look into it in more detail.
Table \ref{comet_sd_pMpAcg_BC_itn_p3.5} contains same exact numbers but grouped differently. In big picture, as average, by looking at bold numbers of table \ref{comet_sd_pMpAcg_BC_itn_p3.5}, experiments well shows that \parlot has better performance on larger input sizes which means longer runs. But for \callgrind it is opposite. For input size B, the average of geomeans of slowdowns is 3.82 and for input C it is 4.88. The key reason of this better performance is more repetition of target data to collect (which is function calls) on larger input sizes, I believe. Even when \parlot gathers system library function calls (pinAll), it has better performance than \callgrind. Figures \ref{comet_chartAvg_sd_B_p3_5} and \ref{comet_chartAvg_sd_C_p3_5} visualizes table \ref{comet_sd_pMpAcg_BC_itn_p3.5} numbers.
\item \textbf{Bandwidth}: Table \ref{comet_bw_pMpAcg_BC_itn_p3.5} shows the required bandwidth for each tool. In addition to big gap between average slowdown of \parlot(main) and \callgrind, \parlot(main) also beats \callgrind in required bandwidth, especially for smaller inputs.(adapted from Related Tools section) Valgrind is shadow value DBI framework (explained in the background section) that maps and records every register and memory value. It gives developers the capability of instrumenting system calls and instructions. Many error detectors such as \textit{Memcheck} have been built on top of Valgrind. \callgrind is a profiling tool  on Valgrind platform that records the call history among functions in a program's run as a call-graph by measuring the number of instructions executed and their relationship to source lines. 
Intermediate generated traces by \callgrind are some numbered files contain pure ascii text. \callgrind enumerate the name of files and function calls and also, stores those numerical values as relative to previous numbers. These are the only data compression options available on \callgrind which is enabled by default. Each \callgrind trace file contains a sequence of function names (or their code) and a few other numbers for each function showing the that function relationships with other functions (caller-callee). There is a tool \textit{callgrind\_ annotate} which displays different reports from the generated traces. From the generated traces by \callgrind in my experiments, the richest report that \textit{callgrind\_ annotate} can produce is the tree of function calls with caller-callee relationship and cost of each function. Cost of each function is the number of Instruction Read which is collected during tracing by reading hardware counters. Cache simulation and branch prediction information also can be enabled to be collected and then \textit{callgrind\_ annotate} can produce different reports for cache and branch prediction. By default, cache simulation and branch prediction (which are originally from another tool Cachegrind) are disabled by \callgrind.
According to table \ref{comet_cr_pMpA_BC_itn_p3.5}, for example for \parlot(all) where the average compression ratio for input C is 644.38, and the correspondent required bandwidth which is 56.38, it shows that \parlot can collect almost 36 MB worth of data per core per second where it only needs 56.38 KB/S bandwidth.]
\end{itemize}

\begin{figure}[!t]
\centering
\includegraphics[width=3.9in]{figs.comet/comet_chartAvg_sd_B_p3_5.png}
\caption{ Input: \textbf{B} - Slowdown of \parlot(main,all) and \callgrind. Each bar is the average slowdown of each tool on each application for 1, 4 and 16 nodes (16, 64 and 256 cores). Last group of bars is GeoMean (from bold numbers in table \ref{comet_sd_pMpAcg_BC_itn_p3.5}). 
}
\label{comet_chartAvg_sd_B_p3_5}
\end{figure}


\begin{figure}[!t]
\centering
\includegraphics[width=3.9in]{figs.comet/comet_chartAvg_sd_C_p3_5.png}
\caption{ Input: \textbf{C} - Slowdown of \parlot(main,all) and \callgrind. Each bar is the average slowdown of each tool on each application for 1, 4 and 16 nodes (16, 64 and 256 cores). Last group of bars is GeoMean (from bold numbers in table \ref{comet_sd_pMpAcg_BC_itn_p3.5}). 
}
\label{comet_chartAvg_sd_C_p3_5}
\end{figure}




\subsection{\parlot Inner Analysis}
	\begin{itemize}
	\item \textbf{Compression Ratio}: Table \ref{comet_cr_pMpA_BC_itn_p3.5} shows the compression ratios for all configs and inputs. On average, \parlot can store up to more than 1700 MB of collected data in just 1 MB trace files. Compression ratios are higher for larger input sizes (reason: repetition of function calls). Also \parlot has better compression performance when it only collects function calls from main application image.
	\item \textbf{Pure \pin Overhead}: Tables \ref{comet_wo_det_Main_all_B_p3.5}, \ref{comet_wo_det_All_all_B_p3.5}, \ref{comet_wo_det_Main_all_C_p3.5} and  \ref{comet_wo_det_All_all_C_p3.5} show average overhead added to each application by different variations of \parlot. Reason of these experiments is to show the \textbf{impact of our data compression approach} and \textbf{pure overhead added by \pin}. \textbf{npin} is just the slowdown caused by initializing \pin's routines on top of the target application without doing anything else (no instrumentation, tracing, compression). In \textbf{wpin}, all collected data would be stored as is to the disk (tracing without compression) (fig \ref{overviewAll}). Last row of tables shows geometric mean of each of its above values showing how much each phase of ParLOT slows down the native execution. In general, we all expect that the slowdowns of $npin < ParLOT < wpin $. But majority of numbers are not like that. Highlighted cells in tables are the ones which do not follow the above order. Number of highlighted cells for \parlot(main) is larger than \parlot(all). Figures \ref{comet_chartAvg_serr_B_p3_5}, \ref{comet_chartAvg_serr_C_p3_5}, \ref{comet_chartAvg_var_B_p3_5} and  \ref{comet_chartAvg_var_C_p3_5} which are the standard errors and variances of runtimes, explain the reason of these highlighted cells. Variability of runtimes for \parlot(main) is more than other ones (native run, \parlot(all) and \callgrind). Maybe if I put the maximum or average of runtimes in the tables, the number of highlighted cells decreases.\\
	Figures \ref{comet_chartDet_B_wc_byTool_p3_5}, \ref{comet_chartDet_C_wc_byTool_p3_5}, \ref{comet_chartDet_B_woc_byTool_p3_5} and \ref{comet_chartDet_C_woc_byTool_p3_5} clearly show the performance of \parlot and impact of \parlot 's compression mechanism. (more explanations in chart caption of figure \ref{comet_chartDet_B_wc_byTool_p3_5} and \ref{comet_chartDet_B_woc_byTool_p3_5})
	\item \textbf{Impact of Compression} - Figures \ref{comet_chartDet_B_wc_byTool_p3_5}, \ref{comet_chartDet_C_wc_byTool_p3_5}, \ref{comet_chartDet_B_woc_byTool_p3_5} and \ref{comet_chartDet_C_woc_byTool_p3_5}
	\item \textbf{Variability} - Figures \ref{comet_chartAvg_serr_B_p3_5}, \ref{comet_chartAvg_serr_C_p3_5}, \ref{comet_chartAvg_var_B_p3_5} and  \ref{comet_chartAvg_var_C_p3_5}.
	\end{itemize}


\begin{figure}[!t]
\centering
\includegraphics[width=2in]{overview-all.png}
\caption{ Overview of \parlot}
\label{overviewAll}
\end{figure}

%%%%%%%%%%%%%%%%%%%%%%%%%%%%%%%%%%%%%
% Bandwidth
%%%%%%%%%%%%%%%%%%%%%%%%%%%%%%%%%%%%%

%%%%%%%%%%%%%%%%%%%%%%%
% Compression Ratio
%%%%%%%%%%%%%%%%%%%%%%%



\input{tabs.comet/comet_bw_pMpAcg_BC_itn_p3.5.tex}

\begin{figure}[!t]
\centering
\includegraphics[width=3.5in]{figs.comet/comet_chartAvg_bw_B_p3_5.png}
\caption{ Input: \textbf{B} - Required Bandwidth per core (KB/s)
}
\label{comet_chartAvg_bw_B_p3_5}
\end{figure}

\begin{figure}[!t]
\centering
\includegraphics[width=3.5in]{figs.comet/comet_chartAvg_bw_C_p3_5.png}
\caption{ Input: \textbf{C}  - Required Bandwidth per core (KB/s)
}
\label{comet_chartAvg_bw_C_p3_5}
\end{figure}



%%%%%%%%%%%%%%%%%%%%%%%
% Detail runtimes
%%%%%%%%%%%%%%%%%%%%%%%

\input{tabs.comet/comet_cr_pMpA_BC_itn_p3.5.tex}


\begin{figure}[!t]
\centering
\includegraphics[width=3.5in]{figs.comet/comet_chartAvg_cr_C_p3_5.png}
\caption{ Input: \textbf{C}  - Compression Ratio
}
\label{comet_chartAvg_cr_C_p3_5}
\end{figure}


\begin{figure}[!t]
\centering
\includegraphics[width=3.5in]{figs.comet/comet_chartAvg_cr_B_p3_5.png}
\caption{ Input: \textbf{B}  - Compression Ratio
}
\label{comet_chartAvg_cr_B_p3_5}
\end{figure}



\begin{table*}[]
\caption{Tracing overhead added by each version of \parlot - Input: B}
\begin{center}
\label{comet_wo_det_Main_all_B_p3.5}
\scalebox{0.80}{
\begin{tabular}{|c|c|rrr|rrr|rrr|rrr|} 
\hline 
\multicolumn{1}{|l|}{\multirow{2}{*}{\textbf{Input: B}}} & \multicolumn{1}{r|}{Nodes :}    & \multicolumn{3}{c|}{1}  & \multicolumn{3}{c|}{4} & \multicolumn{3}{c|}{16}  & \multicolumn{3}{c|}{64} \\ \cline{2-14} 
\multicolumn{1}{|l|}{} & \multicolumn{1}{r|}{Detail Tools:} & \multicolumn{1}{c}{\pininit} & \multicolumn{1}{c}{\parlot} & \multicolumn{1}{c|}{\parlotnc} & \multicolumn{1}{c}{\pininit} & \multicolumn{1}{c}{\parlot} & \multicolumn{1}{c|}{\parlotnc} & \multicolumn{1}{c}{\pininit} & \multicolumn{1}{c}{\parlot} & \multicolumn{1}{c|}{\parlotnc} & \multicolumn{1}{c}{\pininit} & \multicolumn{1}{c}{\parlot} & \multicolumn{1}{c|}{\parlotnc} \\
\hline
\multirow{9}{*}{Main} &  bt  &  1.50  &  1.55  &   5.62  &  1.74  &  1.76  &  5.06  &  2.19  & \cellcolor{blue!25} 2.15  &  5.02  &  1.83  &  2.10  &  3.52 \\
 &  cg  &  1.75  &  1.82  &   2.38  &  1.84  &  1.85  &  2.64  &  2.70  & \cellcolor{blue!25} 2.58  &  4.43  &  2.32  & \cellcolor{blue!25} 2.17  &  4.64 \\
 &  ep  &  2.96  & \cellcolor{blue!25} 2.62  &  20.48  &  1.99  & \cellcolor{blue!25} 1.89  &  5.38  &  2.47  & \cellcolor{blue!25} 1.99  &  3.09  &  2.68  & \cellcolor{blue!25} 2.39  &  2.66 \\
 &  ft  &  1.87  &  2.11  &   6.17  &  1.75  & \cellcolor{blue!25} 1.74  &  2.79  &  2.08  & \cellcolor{blue!25} 1.89  &  2.24  &  2.18  & \cellcolor{blue!25} 1.96  &  2.14 \\
 &  is  &  2.47  &  2.47  &   4.82  &  1.79  & \cellcolor{blue!25} 1.78  &  2.07  &  2.11  & \cellcolor{blue!25} 1.78  &  1.87  &  4.51  & \cellcolor{blue!25} 4.31  &  5.71 \\
 &  lu  &  1.32  & \cellcolor{blue!25} 1.31  &   1.44  &  1.75  &  1.77  &  2.25  &  2.73  &  2.73  &  3.62  &  3.05  &  4.39  &  6.13 \\
 &  mg  &  2.56  & \cellcolor{blue!25} 2.53  &   2.79  &  1.56  & \cellcolor{blue!25} 1.52  &  1.59  &  2.63  & \cellcolor{blue!25} 2.43  &  2.65  &  1.95  &  1.97  &  1.85 \\
 &  sp  &  1.34  & \cellcolor{blue!25} 1.33  &   2.43  &  1.73  &  1.73  &  3.58  &  2.14  &  2.15  &  2.37  &  1.95  &  2.07  &  2.54 \\
\cline{2-14}
 &  GM  &  1.89  &  1.90  &   4.10  &  1.77  & \cellcolor{blue!25} 1.75  &  2.92  &  2.37  & \cellcolor{blue!25} 2.19  &  3.00  &  2.45  &  2.52  &  3.33 \\
\hline 
\end{tabular} }

\end{center}
\end{table*}


\begin{table*}[]
\caption{Tracing overhead added by each version of \parlot - Input: B}
\begin{center}
\label{comet_wo_det_All_all_B_p3.5}
\scalebox{0.8}{
\begin{tabular}{|c|c|rrr|rrr|rrr|rrr|} 
\hline 
\multicolumn{1}{|l|}{\multirow{2}{*}{\textbf{Input: B}}} & \multicolumn{1}{r|}{Nodes :}    & \multicolumn{3}{c|}{1}  & \multicolumn{3}{c|}{4} & \multicolumn{3}{c|}{16}  & \multicolumn{3}{c|}{64} \\ \cline{2-14} 
\multicolumn{1}{|l|}{} & \multicolumn{1}{r|}{Detail Tools:} & \multicolumn{1}{c}{\pininit} & \multicolumn{1}{c}{\parlot} & \multicolumn{1}{c|}{\parlotnc} & \multicolumn{1}{c}{\pininit} & \multicolumn{1}{c}{\parlot} & \multicolumn{1}{c|}{\parlotnc} & \multicolumn{1}{c}{\pininit} & \multicolumn{1}{c}{\parlot} & \multicolumn{1}{c|}{\parlotnc} & \multicolumn{1}{c}{\pininit} & \multicolumn{1}{c}{\parlot} & \multicolumn{1}{c|}{\parlotnc} \\
\hline
\multirow{9}{*}{All} &  bt  &  1.76  &  1.84  &   6.11  &  2.39  &  2.57  &  6.11  &  3.22  &  3.52  &   9.02  &  2.87  &  3.14  &   7.55 \\
 &  cg  &  2.69  &  2.73  &   3.80  &  2.86  &  3.06  &  4.48  &  4.07  &  4.20  &  11.38  &  3.33  & \cellcolor{blue!25} 3.26  &  10.39 \\
 &  ep  &  4.36  & \cellcolor{blue!25} 4.18  &  22.20  &  3.14  &  3.41  &  7.16  &  3.12  &  3.39  &   4.55  &  4.18  & \cellcolor{blue!25} 3.83  &   4.16 \\
 &  ft  &  2.80  & \cellcolor{blue!25} 2.78  &   6.85  &  2.65  &  2.77  &  3.82  &  2.82  &  2.94  &   3.66  &  3.15  & \cellcolor{blue!25} 3.02  &   3.57 \\
 &  is  &  4.40  & \cellcolor{blue!25} 4.22  &   7.04  &  2.85  &  2.96  &  3.42  &  2.91  & \cellcolor{blue!25} 2.83  &   3.24  &  5.38  &  5.44  &   8.81 \\
 &  lu  &  1.70  &  1.73  &   2.39  &  2.54  &  2.76  &  4.88  &  3.96  &  4.30  &  10.47  &  4.45  &  4.65  &  23.41 \\
 &  mg  &  4.83  & \cellcolor{blue!25} 4.75  &   5.37  &  2.51  &  2.79  &  3.07  &  4.32  &  4.46  &   5.22  &  2.73  &  3.17  &   3.26 \\
 &  sp  &  1.70  &  1.72  &   3.01  &  2.46  &  2.66  &  5.06  &  3.27  &  3.65  &   5.67  &  2.77  &  3.31  &  11.65 \\
\cline{2-14}
 &  GM  &  2.78  & \cellcolor{blue!25} 2.77  &   5.59  &  2.66  &  2.86  &  4.58  &  3.42  &  3.62  &   6.02  &  3.51  &  3.65  &   7.41 \\
\hline 
\end{tabular} }

\end{center}
\end{table*}



%\input{tabs.comet/comet_wo_det_Main_all_C_p3.5.tex}

%\input{tabs.comet/comet_wo_det_All_all_C_p3.5.tex}



\begin{figure}[!t]
\centering
\includegraphics[width=4in]{figs.comet/comet_chartDet_B_wc_byTool_p3_5.png}
\caption{ Input: \textbf{B} - This figure and figure \ref{comet_chartDet_C_wc_byTool_p3_5} shows how much of the overhead of \parlot is caused by \pin and its initialization and how much by that section of \parlot that collects traces and compress them. It seems that overhead added by pure \pin does not scale well and increases with growing number of cores.
}
\label{comet_chartDet_B_wc_byTool_p3_5}
\end{figure}


%Figure c wc
\begin{figure}[!t]
\centering
\includegraphics[width=4in]{figs.comet/comet_chartDet_C_wc_byTool_p3_5.png}
\caption{ Input: \textbf{C}
}
\label{comet_chartDet_C_wc_byTool_p3_5}
\end{figure}





\begin{figure}[!t]
\centering
\includegraphics[width=4in]{figs.comet/comet_chartDet_B_woc_byTool_p3_5.png}
\caption{ Input: \textbf{B}- This figure and figure \ref{comet_chartDet_C_woc_byTool_p3_5} shows the impact of \parlot 's data compression. By looking at and comparing green bars of figure \ref{comet_chartDet_B_wc_byTool_p3_5} and \ref{comet_chartDet_B_woc_byTool_p3_5}, clearly it is obvious that how much compressing data improves the performance.
}
\label{comet_chartDet_B_woc_byTool_p3_5}
\end{figure}

\begin{figure}[!t]
\centering
\includegraphics[width=4in]{figs.comet/comet_chartDet_C_woc_byTool_p3_5.png}
\caption{ Input: \textbf{C}
}
\label{comet_chartDet_C_woc_byTool_p3_5}
\end{figure}








\begin{figure}[!t]
\centering
\includegraphics[width=4in]{figs.comet/comet_chartAvg_serr_B_p3_5.png}
\caption{ Input: \textbf{B}  - Standard Error of 3 Runtimes
}
\label{comet_chartAvg_serr_B_p3_5}
\end{figure}




\begin{figure}[!t]
\centering
\includegraphics[width=4in]{figs.comet/comet_chartAvg_serr_C_p3_5.png}
\caption{ Input: \textbf{C}  - Standard Error of 3 Runtimes
}
\label{comet_chartAvg_serr_C_p3_5}
\end{figure}





\begin{figure}[!t]
\centering
\includegraphics[width=4in]{figs.comet/comet_chartAvg_var_B_p3_5.png}
\caption{ Input: \textbf{B}  - Variance of 3 Runtimes
}
\label{comet_chartAvg_var_B_p3_5}
\end{figure}


\begin{figure}[!t]
\centering
\includegraphics[width=4in]{figs.comet/comet_chartAvg_var_C_p3_5.png}
\caption{ Input: \textbf{C}  - Variance of 3 Runtimes
}
\label{comet_chartAvg_var_C_p3_5}
\end{figure}

%\input{pscCometResults.tex}
%\input{pscCometLs5Results.tex}
%\input{pscResults.tex}


    

\section{Summary, Conclusion and Future Work}
\subsection{Summary and Conclusion}
\subsection{Future Work}




\section{Appendix}



   




% An example of a floating figure using the graphicx package.
% Note that \label must occur AFTER (or within) \caption.
% For figures, \caption should occur after the \includegraphics.
% Note that IEEEtran v1.7 and later has special internal code that
% is designed to preserve the operation of \label within \caption
% even when the captionsoff option is in effect. However, because
% of issues like this, it may be the safest practice to put all your
% \label just after \caption rather than within \caption{}.
%
% Reminder: the "draftcls" or "draftclsnofoot", not "draft", class
% option should be used if it is desired that the figures are to be
% displayed while in draft mode.
%
%\begin{figure}[!t]
%\centering
%\includegraphics[width=2.5in]{myfigure}
% where an .eps filename suffix will be assumed under latex, 
% and a .pdf suffix will be assumed for pdflatex; or what has been declared
% via \DeclareGraphicsExtensions.
%\caption{Simulation Results}
%\label{fig_sim}
%\end{figure}

% Note that IEEE typically puts floats only at the top, even when this
% results in a large percentage of a column being occupied by floats.


% An example of a double column floating figure using two subfigures.
% (The subfig.sty package must be loaded for this to work.)
% The subfigure \label commands are set within each subfloat command, the
% \label for the overall figure must come after \caption.
% \hfil must be used as a separator to get equal spacing.
% The subfigure.sty package works much the same way, except \subfigure is
% used instead of \subfloat.
%
%\begin{figure*}[!t]
%\centerline{\subfloat[Case I]\includegraphics[width=2.5in]{subfigcase1}%
%\label{fig_first_case}}
%\hfil
%\subfloat[Case II]{\includegraphics[width=2.5in]{subfigcase2}%
%\label{fig_second_case}}}
%\caption{Simulation results}
%\label{fig_sim}
%\end{figure*}
%
% Note that often IEEE papers with subfigures do not employ subfigure
% captions (using the optional argument to \subfloat), but instead will
% reference/describe all of them (a), (b), etc., within the main caption.


% An example of a floating table. Note that, for IEEE style tables, the 
% \caption command should come BEFORE the table. Table text will default to
% \footnotesize as IEEE normally uses this smaller font for tables.
% The \label must come after \caption as always.
%
%\begin{table}[!t]
%% increase table row spacing, adjust to taste
%\renewcommand{\arraystretch}{1.3}
% if using array.sty, it might be a good idea to tweak the value of
% \extrarowheight as needed to properly center the text within the cells
%\caption{An Example of a Table}
%\label{table_example}
%\centering
%% Some packages, such as MDW tools, offer better commands for making tables
%% than the plain LaTeX2e tabular which is used here.
%\begin{tabular}{|c||c|}
%\hline
%One & Two\\
%\hline
%Three & Four\\
%\hline
%\end{tabular}
%\end{table}


% Note that IEEE does not put floats in the very first column - or typically
% anywhere on the first page for that matter. Also, in-text middle ("here")
% positioning is not used. Most IEEE journals/conferences use top floats
% exclusively. Note that, LaTeX2e, unlike IEEE journals/conferences, places
% footnotes above bottom floats. This can be corrected via the \fnbelowfloat
% command of the stfloats package.



% conference papers do not normally have an appendix


% use section* for acknowledgement
\section*{Acknowledgment}

...


% trigger a \newpage just before the given reference
% number - used to balance the columns on the last page
% adjust value as needed - may need to be readjusted if
% the document is modified later
%\IEEEtriggeratref{8}
% The "triggered" command can be changed if desired:
%\IEEEtriggercmd{\enlargethispage{-5in}}

% references section

% can use a bibliography generated by BibTeX as a .bbl file
% BibTeX documentation can be easily obtained at:
% http://www.ctan.org/tex-archive/biblio/bibtex/contrib/doc/
% The IEEEtran BibTeX style support page is at:
% http://www.michaelshell.org/tex/ieeetran/bibtex/
%\bibliographystyle{IEEEtran}
% argument is your BibTeX string definitions and bibliography database(s)
%\bibliography{IEEEabrv,../bib/paper}
%
% <OR> manually copy in the resultant .bbl file
% set second argument of \begin to the number of references
% (used to reserve space for the reference number labels box)
%\begin{thebibliography,10}
%\bibliography{ganesh}{}
%\bibliographystyle{plain}
%\end{thebibliography}

\bibliographystyle{IEEEtran}
\bibliography{bibs}


% that's all folks
\end{document}





