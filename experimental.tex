

\subsection{Experimental Methodology}
ILCS is a scalable framework for running iterative local searches on HPC platforms.
%
Providing a serial CPU or single-GPU code, ILCS then executes this code in parallel between compute nodes (MPI) and within them (OpenMP and multi-GPU).
%
To evaluate the effectiveness of ideas behind DiffTrace, we have manually injected MPI-level and OMP-level bugs to the Traveling Salesman Problem (TSP) implementation on ILCS framework.
%
Note that the GPU-related functionalities of ILCS are out of scope of this paper.
%
\input{tabs/ilcsPseudoCode-compact.tex}

\begin{figure*}[t]
\centering
\includegraphics[width=6in]{figs/parametersTable.png}
\caption{Filters, Attributes and other Parameters used to pre-process ParLOT Traces (PTs)}
\label{tab:parameters}
\end{figure*}



%
Bugs are injected at different locations of the framework.
%
We have collected ParLOT (main image) traces from the execution of ILCS-TSP with 8 MPI processes and 4 OpenMP threads on each process.



Table \ref{tab:parameters} shows different parameters that we can pre-process PTs with. Each combination of these parameters would result in a different concept lattice, thus different JSM and different clusterings. 
%
A table similar to \ref{tab:bug1candidate} is created for each injected bug.
%
Each row of the table is showing the set of parameters used to create JSMs.
%
Then by calculating $ |JSM(buggy) - JSM(bugfree)| $ we are interested to see which PT changes the most after the bug injected and falls into a single cluster.
%
The object(s) in the cluster with the fewest members (below a threshold) are potential candidates of \textit{threads that are manifesting the bug} and the diff(buggy,bug-free) is in our interest to see how does the bug changes its control flow.

\subsection{Case Study: ILCS-TSP}
Here is the ILCS framework pseudo-code. User needs to write \texttt{CPU\_Init()},\texttt{CPU\_Exec()} and \texttt{CPU\_Output()}.
% ILCS Code for later references


Table \ref{tab.ilcsTspBugs} describes the bug that I injected to ILCS-TSP
\input{tabs/tabs.ilcsTspBugs.tex}

\clearpage

\subsubsection{Bug1: Wrong Operation in MPI AllReduce()}
We have injected a bug (row 1 table \ref{tab:ilcsTspBugs}) where \texttt{MPI\_Allreduce()} had been invoked with a wrong operation in one of the processes ($P_{2}$)(\texttt{MPI\_MAX} instead of \texttt{MPI\_MIN}).
%

\textbf{::What is the runtime reaction to this bug:: Program terminated well without any error, crash, hang or throwing any exception. But the results might be corrupted. This might be a silent bug that diffTrace could reveal}

The last row of table \ref{tab:bug1candidate} is telling us that among all combinations of parameters (filters, attributes, etc.) PT 0 (ParLOT trace that belongs to thread 0 of process 0 got impacted the most after we inject the bug.
\input{tabs/tabs.bug1candidate.tex}

\begin{figure}[t]
\centering
\scalebox{0.5}{
\includegraphics[width=6in]{figs/diffNLR/{bug1-1}.pdf}}
\caption{Bug1: diffNLR $PT_{0,0}$ - buggy vs. native}
\label{fig:bug1-1}
\end{figure}

The target MPI\_Allreduce() that we injected the bug to, finds the rank (i.e., process) that has the ``champion'' result among all of ranks using MPI\_MIN operator. Then that champion rank copies its ``champion results'' to a global data-structure and broadcast the ``champion results'' to all other ranks for the next time step.
However, since we changed the MPI\_MIN to MPI\_MAX in only one of the ranks, the true champion rank would get lost, instead a false champion rank (which turned to be rank 0 or $PT_{0,0}$) would broadcast its results as champion in the first time-step, causing a potential wrong answer. 
\clearpage


\subsubsection{Bug2: Wrong Size in MPI AllReduce() (one process)}
We have injected a bug (row 2 table \ref{tab:ilcsTspBugs}) where \texttt{MPI\_Allreduce()} had been invoked with a wrong size. \textbf{::MORE EXPLANATIONS ABOUT WHAT THE BUG IS::}
%

%\textbf{::What is the runtime reaction to this bug:: on node 3 (rank 3 in comm 0): Fatal error in PMPI\_Bcast: Invalid root}

Similar to table \ref{tab:bug1candidate}, the same ranking system tells us to check $PT_{1,0}$ and $PT_{3,0}$. Note that the bug injected to only one process ($P_3$) to have the wrong size.)

\begin{figure}[t]
\centering
\scalebox{0.5}{
\includegraphics[width=6in]{figs/diffNLR/{bug2-1}.pdf}}
\caption{Bug2: diffNLR $PT_{3,0}$ - buggy vs. native}
\label{fig:bug2-1}
\end{figure}


\begin{figure}[t]
\centering
\scalebox{0.5}{
\includegraphics[width=6in]{figs/diffNLR/{bug2-2}.pdf}}
\caption{Bug2: diffNLR $PT_{1,0}$ - buggy vs. native}
\label{fig:bug2-2}
\end{figure}


\textbf{::EXPLANATIONS OF OBSERVATIONS::}

\clearpage


\subsubsection{Bug3: Wrong Size in MPI AllReduce() (all processes)}
We have injected a bug (row 3 table \ref{tab:ilcsTspBugs}) where \texttt{MPI\_Allreduce()} had been invoked with a wrong size. \textbf{::MORE EXPLANATIONS ABOUT WHAT THE BUG IS::}
%

\textbf{::What is the runtime reaction to this bug:: on node 3 (rank 3 in comm 0): Fatal error in PMPI\_Bcast: Invalid root}

Similar to table \ref{tab:bug1candidate}, the same ranking system tells us to check $PT_{1,0}$ and $PT_{3,0}$. Note that the bug injected to only one process ($P_3$) to have the wrong size.)

\begin{figure}[t]
\centering
\scalebox{0.5}{
\includegraphics[width=6in]{figs/diffNLR/{bug2-1}.pdf}}
\caption{Bug2: diffNLR $PT_{3,0}$ - buggy vs. native}
\label{fig:bug2-1}
\end{figure}


\begin{figure}[t]
\centering
\scalebox{0.5}{
\includegraphics[width=6in]{figs/diffNLR/{bug2-2}.pdf}}
%\caption{Bug2: diffNLR $PT_{1,0}$ - buggy vs. native}
\label{fig:bug2-2}
\end{figure}


\textbf{::EXPLANATIONS OF OBSERVATIONS::}

\clearpage

\subsubsection{Bug4: Wrong Op in MPI AllReduce(): no effect!,program terminates fine}

maybe all images show some reflection

\subsubsection{Bug5: Wrong Size in next MPI AllReduce()(one process)::no effect, program terminates fine}

maybe all images show some reflection

\subsubsection{Bug6: Wrong Size in next MPI AllReduce()(all processes)::no effect,program terminates fine}

maybe all images show some reflection

\subsubsection{Bug7: Wrong Size in MPI Bcast()(one process)::}

maybe all images show some reflection

\subsubsection{Bug8: Wrong Size in next MPI Bcast()(all processes)::no effect,program terminates fine}

maybe all images show some reflection

\subsubsection{3: Missing Critical Section one thread in on process}

I planted the bug (missing critical section) in process 2

\begin{figure*}[t]
\centering
\includegraphics[width=6in]{figs/{miscrit-1-1-ranking}.png}
\caption{Part of ranking table for MisCrit 1-1}
\label{miscrit-1-1-rank}
\end{figure*}



\begin{figure*}[t]
\centering
\includegraphics[width=5.9in]{figs/{miscrit-1-1-1.3_2.3}.pdf}
\caption{diffNLR of process 1 thread 3 and process 2 thread 3 buggy(missing critical section) vs. bug-free}
\label{bug2.2}
\end{figure*}


