
\noindent{\bf BEGIN Ganesh}

Debugging high performance computing code
remains a challenge at all levels of scale.
%
Conventional HPC debuggers~\cite{allinea-ddt,roguewave,others}
excel at many tasks such as examining the execution
state of a complex simulation at a detailed
level and allowing the developer to re-execute
the program close to the point of failure.
%
However, they do not provide a good understanding
of why a program version that worked earlier
failed upon upgrade or feature addition.
%
Innovative solutions are needed to highlight the
salient differences between two executions in a manner
that makes debugging easier as well as more systematic.
%
A recent study conducted under the auspices of the
DOE~\cite{DBLP:journals/corr/GopalakrishnanH17}
provides a comprehensive survey
of existing debugging tools.
%
It classifies them under
{\em four} software organizations (serial, multithreaded,
multi-process, and hybrid), {\em six}
method types (formal methods, static analysis, dynamic
analysis, nondeterminism control, anomaly detection,
and parallel debugging), and lists a total of 30 specific
tools.
%
Despite this abundance of tools and approaches, many
significant problems remain to be solved before debugging
{\em can be approached by the HPC community as a collaborative
activity} so that HPC developers can share their solutions
and extend a common framework


In this paper, we provide our fundamentally fresh look
at debugging.
%
We point out three significant problems that
we have addressed in our work, and provide our preliminary
solutions backed up by case studies.
%
While our work has not (yet) addressed the situations in
which millions of threads and thousands of processes run
for days and produce an error,
we strongly believe that we can get there only through
a series of {\em rigorous} approaches
that overcome key limitations found in conventional
debugging approaches in a step-by-step manner, accompanied
by careful measurements of the merits of the new approach.
%
The main contribution of this paper is the first such
critical measurements of our proposed approach.


\paragraph{Problem-1: Need to Generalize Approaches for Outlier Detection:\/}
%
Almost all debugging approaches seek to find outliers (``unexpected
executions'') amongst thousands of running processes and threads.
%
The approach taken by most existing tools is to
look for symptoms in a specific bug-class that they
cover.
%
Unfortunately,
this approach calls for a programmer having a good guess of what
the underlying problem might be,
and to then pick the right set of tools to deploy.
%
If the guess is wrong, the programmer has no choice but to
refine their guess
and look for bugs in another class,
re-executing the application and hoping for
better luck with another tool.
%
This iterative loop of re-execution followed by applying a
best-guess tool for the suspected bug class can potentially consume
large amounts of execution cycles and also waste an
expert developer's time.


\noindent{\em Solution to Problem-1: Whole Program Tracing for Debugging:\/}
In this setting, our first contribution is
a debugging approach in which the application is not merely
run with a single symptom-specific tool attached as described earlier.
%
Instead, we collect {\em whole program traces} of function calls
and returns, using a PIN-based function call tracing facility called
ParLoT that we have developed and previously reported~\cite{parlot-paper}.
%
We store these traces for potential examination by multiple tools
and approaches.
%
The advantage of whole program binary tracing supported
in ParLoT is that we can
collect function calls at {\em any desired level of abstraction}.
%
For instance, if the programmer wants to cover activities at the
MPI level, the OpenMP level and perhaps even lower levels (e.g., the
MPI library or the OpenMP runtime), they can do so using ParLoT.


Clearly, the more APIs at which function calls are recorded, the
more burdensome trace collection becomes.
%
However, the advantage is that correspondingly
more tools can then be applied to
the collected traces.
%
There is always a sweet-spot in this trade-off space,
depending on the particular debugging situation.
%
However, our fundamental insight is that {\em given the
inevitability of heterogeneous programming} (the use of multiple
concurrency models), it is important to
be collecting traces from a few related APIs at a time, so that
one can study bugs in one of the concurrency models or a bug
resulting from a bad cross-model interaction
from a {\em single run of the program}.


In our research, we have thus far demonstrated the advantage of
ParLoT with respect to collecting both MPI and OpenMP traces
from a {\em single run of a hybrid MPI/OpenMP program}.
%
We demonstrate that from this single type of traces, it is possible
to pick out MPI-level bugs or OpenMP-level bugs.
%
While we have not covered all these combinations in our work
so far, the main contribution claimed is the ability to
cover multiple APIs while debugging, and without re-executions or
guess-work.


While this approach to whole-program tracing
may sound extremely computation intensive, we employ
novel on-the-fly compression techniques within ParLoT.
%
In our previous study~\cite{parlot-paper}, we report compression
efficiencies exceeding 16,000.
%
This allows us to bring out the function call traces without
significantly burdening the memory subsystem or I/O networks in the HPC cluster.


\paragraph{Problem-2: Need to Generalize Approaches for Outlier Detection:\/}
Given that outlier detection is central to debugging,
it is important to be employing efficient representations of the traces
collected from threads and processes so that one can compute
{\em distances} between these traces more systematically, without
involving human reasoning in the loop.
%
The representation must also be versatile enough to
be able to ``Diff'' the traces\footnote{Hence the name of our tool, {\bf DiffTrace}.}
with respect to {\em an extensible number of vantage points}.
%
These vantage points could be diffing with respect to process level activities,
diffing with respect to thread-level activities, a combination thereof,
or even finite sequences of process/thread calls (say, to locate {\em changes}
in caller/callee relationships).


\noindent{\em Solution to Problem-2: Use of Concept Lattices in Debugging:\/}
In DiffTrace, we employ {\em concept lattices} to amalgamate the collected traces.
%
Concept lattices have previously been employed in HPC to perform structural
clustering of process behaviors~\cite{weber-cl} to present performance data more
meaningfully to users.
%
The authors of that paper employ the notion of {\em Jaccard distances}
to cluster performance results that are closely related to process structures
(determined based on caller/callee relationships).


In DiffTrace, we employ incremental algorithms for building and maintaining
concept lattices from the ParLoT-collected traces.
%
In addition to Jaccard distances, in our work we also perform hierarchical
clustering of traces and provide a tunable threshold for outlier detection.
%
We believe that these uses of concept lattices and more refinement approaches
for outlier detection are new in HPC debugging.


\paragraph{Problem-3: Loop Detection:\/}
Most programs spend most of their time in loops.
%
Therefore it is important to employ state-of-the-art algorithms for
loop extraction from execution traces.
%
It is also important
to be able to diff two executions with respect to changes in their looping behaviors.

\noindent{\em Solution to Problem-3: Rigorous Approaches to Loop Analysis:\/}
In DiffTrace, we employ the notion of NLRs (what does it stand for?) for
extracting loops.
%
Each repetitive loop structure is given an identifier, and nested loops are
expressed as repetitions of this identifier exponentiated (as with regular
expressions).
%
This approach to summarizing loops can help manifest
bugs where the program does not hang or crash, but nevertheless
run differently in a manner that informs the developer engaged in debugging.


\noindent To summarize, the key contributions of this paper are the following [[fix the section numbers
later]]:

\begin{itemize}
\item A method to organize function call traces collected from processes and
      threads into concept lattices, and a method to
      detect loops from dynamic traces (Section~\ref{sec3}).

\item Details of the algorithms employed in DiffTrace (Section~\ref{sec4}).

\item Experimental studies on a heterogeneous program called
      Iterated Local Champion Search (ILCS, Section~\ref{sec5}).

\item Strengths and limitations of DiffTrace, plans for future work (Section~\ref{sec6}).
\end{itemize}



\noindent{\bf END Ganesh}

\hl{[[Ganesh and Saeed have written some text before for the intro which is available in v0/intro.tex (also available but commented in current file). Current version is based on our discussion on May 8th]]}

\begin{itemize}
	\item Importance of whole program diffing : understand changes, debug (DOE REPORT \cite{hpcdoe})
	\item Efficient tracing supports selective monitoring at multiple levels
	\begin{itemize}
		\item Bugs not there at a predictable API level
		\item Prior work (ParLoT) supports whole program tr.
	\end{itemize}
	\item Dissimilarity is important to know: bugs, changes during porting,...
	\item Key enablers of meaningful diffing:
	\begin{itemize}
		\item Formal concepts (novel contrib to debugging)
		\item Loop detection (loop diffing can help)
	\end{itemize}
	\item Importance, given the growing heterogeneity
\end{itemize}


%When a new version of an HPC software system is created, logical errors often get introduced.
%
%To maintain productivity, designers need effective and efficient methods to locate these errors.
%
%Given the increasing use of hybrid (MPI + X) codes and library functions, errors may be introduced through a usage contract violation at multiple interfaces.
%
%Therefore, tools that can record activities at all involved APIs are necessary.
%
%Designers find most of these bugs manually, and the efficacy of a debugging tool is often measured by how well it can highlight the salient differences between the executions of two versions of software.
%
%Given the huge number of things that could be different -- individual iterative patterns of function calls, groups of functions calls, or even specific instruction types (e.g., non-vectorized versus vectorized floating-point dot vector loops) -- designers cannot afford to rerun the application many times to collect each facet of behavior separately.
%
%These issues are well summarized in many recent studies \cite{hpcdoe}


%One of the major challenges of HPC debugging is the huge diversity of applications, which encompass domains such as computational chemistry, molecular dynamics, and climate simulation.
%
%In addition, there are many types of possible “bugs” or, more precisely, errors. An \textbf{error} may be a deadlock or a resource leak. These errors may be caused by different \textbf{faults}: an unexpected message reordering rule (for a deadlock) or a forgotten free statement (for a resource leak).
%
%There exists a collection of scenarios in which a bug can be introduced: when developing a brand-new application, optimizing an existing application, upscaling an application, porting to a new platform, changing the compiler, or even changing compiler flags.
%
%Unlike traditional software, there are hardly any bug-repositories, collection of trace data or debugging-purpose benchmarks in the HPC community.
%
%The heterogeneous nature of HPC bugs makes developers come up with their own solutions to resolve specific classes of bugs on specific architectures or platforms that are not usable elsewhere \cite{hpcdoe}.

% Why we need always on tracing
%When a failure occurs (e.g., a deadlock or crash) or the application outputs an unexpected result, it is not economic to rerun the application and consume resources to reproduce the failure. Moreover, HPC bugs might not be reproducible due to non-deterministic behavior, which is common in HPC applications. Also, the failure might be caused by a bug present at different APIs, system levels or the network, thus multiple reruns might be needed to locate the bug.

%ParLOT introduction
%In previous work \cite{parlot}, we have introduced ParLOT, a tool that efficiently collects whole program function-call traces using dynamic binary instrumentation.
%
%ParLOT captures function calls and returns at different levels (e.g., application and library code) and incrementally compress them on-the-fly, resulting in low runtime overhead and low tracing bandwidth, preserving the whole-program dynamic behavior for offline analysis.
%

% Post-mortem analysis to obtain understanding about different aspects of the dynamic behavior
%In the current work, we introduce DiffTrace, a tool-chain for the post-mortem analysis of \textit{ParLOT Traces} (PTs) that supplies developers with information about dynamic behavior of HPC applications at different levels with different granularities towards debugging. 
%
%The topology of HPC tasks on both distributed and shared memory often follows a ``symmetric'' flow of control such as SPMD, master/slave, or odd/even where multiple tasks contain \textit{similar} events in their control flow.
%
%HPC bugs often manifest themselves as divergence in the control flow of processes compared to what was expected.
%
%In other words, HPC bugs violate the rule of ``symmetric'' and ``similar'' control flow of one or more threads/processes in typical HPC applications based on the original topology of the application.
%
%We believe that finding the dissimilarities among traces is the essential initial step towards locating the bug and identifying the root cause.

%Large-scale HPC application execution would result in thousands of PTs due to the execution of thousands of processes and threads.
%
%Since HPC applications spend most of their time in an outer main loop, every single PT also may contain sequences of millions of trace entries (i.e., function calls and returns).
%
%Finding the bug manifestation (i.e., dissimilarities caused by the bug) among a large number of long PTs is the problem of finding the needle in the haystack.

%
%Decompressing PTs collected from long-running large-scale HPC applications for offline analysis produces an overwhelming amount of data. However, missing any piece of collected data may result in losing key information about the application behavior.
%
%We propose a variation of the NLR (Nested Loop Recognition) algorithm \cite{Ketterlin-nlr} that takes a sequence of trace entries as input and, by recursively detecting repetitive patterns, re-compresses traces into ``iterative sets'' in a lossless fashion (intra-PT compression).  
%

%Analyzing the application execution as a whole (inter-PT compression) is another goal that we are pursuing in this work. 
%
%By extracting \textit{attributes} from pre-processed traces, we inject them into a concept hierarchy data structure called Concept Lattice \cite{clbook}.  
%
%Concept lattices give us the capability to reduce the search space from thousands of instances to just a few \textit{equivalence classes of traces} by measuring the similarity of traces \cite{Alqadah2011}, making the process of finding the needle in the haystack more feasible.
%
%Comparison of the bug-free concept lattice and its equivalent classes with the buggy version of the same application reveals insights about the dynamic behavior of the program and how the bug changed the classes and their members.
%

%Fowlkes et al.~\cite{fowlkes83} proposed a method for comparing two hierarchical clusterings by counting the objects that fall into the same or different clusters.
%
%Inspired by Fowlkes's approach, we believe that the PTs that fall into different classes before and after the bug are the potential PTs that manifest the bug impact and/or reflect the bug's root cause.
%
%These candidate PTs then require deeper \textit{observing} and \textit{diffing} with their corresponding bug-free PTs to see what has been changed when the bug was introduced.
%
\hl{**   TODO: 
Highlights of results obtained as a result of the above thinking should be here. This typically comes before ROADMAP of paper.}

In summary, this paper makes the following main contributions:
\begin{itemize}
\item A tunable tracing and trace-analysis tool-chain for HPC application program understanding and debugging
\item A variation of the NLR algorithm to compress traces in lossless fashion for easier analysis and detecting (broken) loop structures
\item An FCA-based clustering approach to efficiently classify traces with similar behavior
\item A tunable ranking mechanism to highlight suspicious trace instances for deeper study
\item A visualization framework that reflects the points of differences or divergence in a pair of sequences.
\end{itemize}

%
\hl{
The rest of the paper is as follows:

- Sec 2: Background

- Sec 3: Components

- Sec 4: Case Study: ILCS

- Sec 5: Related Work

- Sec 6: Concluding Remarks
}
