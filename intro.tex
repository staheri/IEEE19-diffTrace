The act of rendering any software design into code invariably
introduces bugs.
%
The code is tested till it stops readily revealing bugs, and then it is
put into production.
%
The remaining lurking bugs often remain dormant, but
rear their head when the inputs are changed or the code
is ported to new platforms.
%
While traditional software engineering often achieves high
quality control, these methods are largely inapplicable to HPC
where concurrency combined with problem-scale and the
presence of very sophisticated domain-specific math
makes programming really hard.
%
Good quality HPC software almost always requires extremely skilled
developers.
%
Unfortunately, good developers are rare, and the need to deploy HPC software
only keeps growing, thus putting a lot of burden on the shoulders of those
who still are maturing.
%
The only solution is that debugging tools must begin doing far more heavy lifting
than traditionally expected.


In HPC, bugs are a function of both flawed program logic as well as unspecified
and illegal interactions between various concurrency models ({\em e.g.}, PThreads,
MPI, OpenMP, etc.) that coexist in any
large application.
%
The best hope for debugging in many cases
lies in being able to efficiently capture detailed execution
traces and compare them against traces emerging from previous (and more stable) 
software versions.
%
This paper offers ParLot, an efficient binary-level tracing tool that 
(1)~traces compiled applications with very little overhead, (2)~employs
advanced data compression methods to bring out traces across the memory
hierarchy with dramatically reduced bandwidth needs, and (3)~provides debuggers
(humans and debugging tools) with traces that capture many valuable pieces of
information such as function call/return chains, function call frequencies, and
the total number of various instruction types executed.
%
The shocking thing is that despite being at the threshold of exascale computing,
the community does not have a tracing tool that has anywhere near the level
of efficacy that ParLot has.


This paper itself does not propose debugging methods.
%
Instead, the purpose of this paper is to really take stock of what else is
available for tracing and debugging large-scale applications, and perform 
a fairly thorough evaluation of their capabilities vis-a-vis ParLot.
%
Thus, we aim to scientifically establish that we indeed have solved the
tracing problem better than anyone else has.
%
We then appeal to past work (some of it being our own past successes) showing that we now are in
a position to breathe new life into such tracing-based debugging tools, now that
ParLot will be able to efficiently provide trace information to them.

Here are the main contribution of this paper:
\begin{itemize}
\item Tracing
\item Compression
\item Stack correction
\end{itemize}

The remainder of this paper is organized as follows. In section
\ref{sec:background}, we introduce the basic ideas and infrastructures behind ParLot. Section \ref{sec:reltools} describes existing tracing tools and techniques and their relevance
to our approach. We present the implementation of Parlot in Section \ref{sec:impl}. Section \ref{sec:evalmeth} and \ref{sec:results} shows our evaluation of different aspects of Parlot and its comparsion with \textit{Callgrind}.
