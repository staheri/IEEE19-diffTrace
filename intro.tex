Debugging high-performance computing code
remains a challenge at all levels of scale.
%
Conventional HPC debuggers~\cite{ddt,totalview}
excel at many tasks such as examining the execution
state of a complex simulation in detail
and allowing the developer to re-execute
the program close to the point of failure.
%
However, they do not provide a good understanding
of why a program version that worked earlier
failed upon upgrade or feature addition.
%
Innovative solutions are needed to highlight the
salient differences between two executions in a manner
that makes debugging easier as well as more systematic.
%
A recent study conducted under the auspices of the
DOE~\cite{hpcdoe}
provides a comprehensive survey
of existing debugging tools.
%
It classifies them under
four software organizations (serial, multithreaded,
multi-process, and hybrid), six
method types (formal methods, static analysis, dynamic
analysis, nondeterminism control, anomaly detection,
and parallel debugging), and lists a total of 30 specific
tools.
%
Despite this abundance of activity and tools, many
significant problems remain to be solved before debugging
{\em can be approached by the HPC community as a collaborative
activity} so that HPC developers can extend a common
framework.


Almost all debugging approaches seek to find outliers (``unexpected
executions'') amongst thousands of running processes and threads.
%
The approach taken by most existing tools is to
look for symptoms in a specific bug-class that they cover.
%
Unfortunately,
this approach calls for a programmer having a good guess of what
the underlying problem might be,
and to then pick the right set of tools to deploy.
%
If the guess is wrong, the programmer has no choice but to
refine their guess
and look for bugs in another class,
re-executing the application and hoping for
better luck with another tool.
%
This iterative loop of re-execution followed by applying a
best-guess tool for the suspected bug class can potentially consume
large amounts of execution cycles and wastes an
expert developer's time.
%
More glaring is the fact that these tools must recreate the
execution traces yet again: they do not have means to hand off
these traces to another tool or cooperate in symbiotic ways.



We cannot collect all relevant pieces of information
necessary to detect all possible bug classes such as
resource leaks, deadlocks, and data races.
%
Each such bug requires its attributes to be kept.
%
Also, debugging is not fully automatable (it is
an undecidable problem in general) and must involve human thinking:
at least to reconcile what is observed against the deeper application-level semantics.
%
However, (1)~we believe that it is still possible to collect one standard set
of data and use it to make an initial triage in such
a way that it can guide a later, deeper debugging phase to locate
which of the finer bug gradations (e.g., resource leaks or races) brought
the application down.
%
Also, (2)~we believe that it is possible to engage the human {\em with respect
to understanding structured presentations of information}.



Our DiffTrace framework addresses both issues.
%
The common set of data it uses is a {\em whole program
function call trace} collected per process/thread.
%
DiffTrace relies on
novel ways to diff a normal trace and a fault-laden trace to guide the
debugging engineer closer to the bug.
%
While our work has not (yet) addressed situations in
which millions of threads and thousands of processes run
for days before they produce an error,
we strongly believe that we can get there
once we understand the pros and cons of our initial
implementation of the DiffTrace tool, which is described in this paper.
%
The second issue is handled in DiffTrace by offering a novel
collection of modalities for understanding program execution diffs.
%
We now elaborate on these points by addressing the following three problems.



\paragraph{Problem 1 -- Collecting Whole-Program Heterogeneous Function-Call Traces
Efficiently\/} Not only must we have the ability
to record function calls and returns at one
API such as MPI, increasingly we must collect calls/returns at multiple
interfaces (e.g., OpenMP, PThreads, and even inner levels such as TCP).
%
The growing use of heterogeneous parallelization necessitates that we 
understand MPI and OpenMP activities (for example) to locate cross-API
bugs that are often missed by other tools.
%
Sometimes, these APIs contain the actual error (as opposed to the user code), and it would be attractive to have this debugging ability.


{\em Solution to Problem 1:\/}
In DiffTrace, we choose Pin-based whole program binary tracing, with
tracing filters that allow the designer to collect a suitable mixture of API
calls/returns.
%
We realize this facility using
ParLOT, a tool designed by us and published earlier~\cite{parlot}.
%
In our research, we have thus far demonstrated the advantage of
ParLOT with respect to collecting both MPI and OpenMP traces
from a {\em single run of a hybrid MPI/OpenMP program}.
%
We demonstrate that, from this single type of trace, it is possible
to pick out MPI-level bugs and/or OpenMP-level bugs.
%
While whole-program tracing
may sound extremely computation and storage intensive, ParLOT employs
lightweight on-the-fly compression techniques to keep these overheads low.
%
It achieves compression ratios exceeding 21,000~\cite{parlot},
thus making this approach practical, demanding
only a few kilobytes per second per core of bandwidth.


\paragraph{Problem 2 -- Need to Generalize Techniques for Outlier Detection\/}
Given that outlier detection is central to debugging,
it is essential to use efficient representations of the traces
to be able to systematically compute
{\em distances} between them without
involving human reasoning.
%
The representation must also be versatile enough to
be able to ``diff'' the traces
with respect to {\em an extensible number of vantage points}.
%
These vantage points could be diffing concerning process-level activities,
thread-level activities, a combination thereof,
or even finite sequences of process/thread calls (say, to locate {\em changes}
in caller/callee relationships).


{\em Solution to Problem 2:\/}
DiffTrace employs {\em concept lattices} to amalgamate the collected traces.
%
Concept lattices have previously been employed in HPC to perform structural
clustering of process behaviors~\cite{weberStructural} to present performance data more
meaningfully to users.
%
The authors of that paper use the notion of {\em Jaccard distances}
to cluster performance results that are closely related to process structures
(determined based on caller/callee relationships).
%
In DiffTrace, we employ incremental algorithms for building and maintaining
concept lattices from the ParLoT-collected traces.
%
In addition to Jaccard distances, in our work, we also perform hierarchical
clustering of traces and provide a tunable threshold for outlier detection.
%
We believe that these uses of concept lattices and refinement approaches
for outlier detection are new in HPC debugging.


\paragraph{Problem 3 -- Loop Summarization\/}
Most programs spend most of their time in loops.
%
Therefore, it is important to employ state-of-the-art algorithms for
loop extraction from execution traces.
%
It is also important
to be able to diff two executions with respect to changes in their looping behaviors.
%
In our experience, presenting such changes using good visual metaphors
tends to highlight many bug types immediately.


{\em Solution to Problem 3:\/}
DiffTrace utilizes the rigorous notion of Nested Loop Representations (NLRs) for
extracting loops.
%
Each repetitive loop structure is given an identifier, and nested loops are
expressed as repetitions of this identifier exponentiated (as with regular
expressions).
%
This approach to summarizing loops can help manifest
bugs where the program does not hang or crash but nevertheless
runs differently in a manner that informs the developer engaged in debugging.

{\bf Organization\/}: 
%
\S\ref{sec:overview} illustrates the contributions of this paper on a simple example.
%
\S\ref{sec:algo} presents the algorithms underlying DiffTrace in more detail.
%
\S\ref{sec:ilcs-case-study} summaries the experimental methodology before showing a medium-sized case study involving MPI and OpenMP.
%
\S\ref{sec:lulesh} shows initial measurements and examples on LULESH~\cite{LULESH2:changes}, a DOE common mini app.
%
\S\ref{sec:related} summarizes selected related works.
%
\S\ref{sec:discussion} concludes the paper with a discussion.

% 
% \Noindent To summarize, the key contributions of this paper are the following
% \begin{itemize}
% \item A method to organize function call traces collected from processes and
%       threads into concept lattices, and a method to
%       detect loops from dynamic traces (Section~\ref{sec3}).
% 
% \item Details of the algorithms employed in DiffTrace (Section~\ref{sec4}).
% 
% \item Experimental studies on a heterogeneous program called
%       Iterated Local Champion Search (ILCS, Section~\ref{sec5}).
% 
% \item Strengths and limitations of DiffTrace, plans for future work (Section~\ref{sec6}).
% \end{itemize}
% 





