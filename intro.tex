
Detecting and root causing HPC bugs is expensive. While traditional software engineering generally achieves high quality control, these methods are often inapplicable to HPC where concurrency combined with large problem scales and sophisticated domain-specific math can make programming extremely challenging. For example, it took months for scientists to debug an MPI laser-plasma interaction code~\cite{hpcdoe}.

HPC bugs are typically a function of both flawed program logic as well as unspecified or illegal interactions between various concurrency models ({\em e.g.}, PThreads, MPI, OpenMP, etc.) that coexist in most large applications. Moreover, HPC software tends to consume enormous amounts of CPU time and hardware resources. Reproducing bugs by rerunning the application in case of execution failure is therefore expensive, time consuming, and inefficient. The best hope for debugging lies in being able to efficiently capture detailed execution traces and compare them against traces from previous (stable) runs~\cite{cstg,stat}. However, collecting informative data during execution may require source-code modifications and recompilation or may add too much overhead to the runtime. Thus the need is great for a generic low-overhead tracing tool that collects effective execution information dynamically and that is portable to multiple parallel platforms.

This paper presents \parlot, an efficient binary-level tracing tool that captures traces replete with debugging information to help locate a variety of possible bug types through offline trace analysis (without needing application reruns) if/when the application runs into an error. The mindset used here is \textit{``pay a little upfront to dramatically reduce the number of overall debug iterations''}. This paper makes the following main contributions.
\begin{itemize}
\item It introduces a new tracing approach that makes it possible to capture the full call-return, call-stack, call-graph, and call-frequency information, including all library calls, for every thread and process of even large-scale applications at low overhead in both space and time.
\item It describes advanced data compression methods to drastically reduce the required tracing bandwidth, thus enabling the collection of a rich set of information, which would be infeasible without on-the-fly compression.
\item It presents \parlot, a proof-of-concept tool that implements our compression-based low-overhead tracing approach. \parlot is capable of instrumenting any x86-based application at the binary level (regardless of the source language used) to collect its full function-call trace.
\end{itemize}
The remainder of this paper is organized as follows. Section \ref{sec:bgreltool} introduces the basic ideas and infrastructure behind \parlot and other tracing tools. Section \ref{sec:design} describes the design of \parlot in detail. Sections \ref{sec:evalmeth} and \ref{sec:results} present our evaluation of different aspects of \parlot and compare it with a similar tool. Section \ref{sec:???} concludes the paper with a summary and future work.


\subsection{Key strong points of \parlot from results}

\begin{itemize}
\item Low Tracing overhead - Table \ref{comet_sd_pMpAcg_BC_itn_p3.5} - Fig \ref{comet_chartAvg_sd_B_p3_5}, \ref{comet_chartAvg_sd_C_p3_5} - Section \ref{subsec:lowtoh}
\item Very low required bandwidth - Table \ref{comet_bw_pMpAcg_BC_itn_p3.5} - Fig \ref{comet_chartAvg_bw_B_p3_5}, \ref{comet_chartAvg_bw_C_p3_5} - Section \ref{subsec:lowbw}
\item Enormous amount of rich data can be reproduced from compressed trace sizes with very low required bandwidth - Table \ref{comet_cr_pMpA_BC_itn_p3.5} - Fig \ref{comet_chartAvg_cr_B_p3_5}, \ref{comet_chartAvg_cr_C_p3_5} - Section \ref{subsec:cr}
\item Majority of the overhead caused by Pin\_ init - Table \ref{comet_wo_det_All_all_B_p3.5}, \ref{comet_wo_det_Main_all_B_p3.5} - Fig \ref{comet_chartDet_B_wc_byTool_p3_5}, \ref{comet_chartDet_C_wc_byTool_p3_5} - Section: \ref{subsec:pinit}
\item \parlot without compression is terrible. (high impact of compression method on performance) - Table \ref{comet_wo_det_All_all_B_p3.5}, \ref{comet_wo_det_Main_all_B_p3.5}  - Fig \ref{comet_chartDet_B_woc_byTool_p3_5}, \ref{comet_chartDet_C_woc_byTool_p3_5} - \cite{subsec:compact}
\end{itemize}


