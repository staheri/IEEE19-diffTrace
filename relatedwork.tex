Three major recent studies
have emphasized the need for better debugging tools
{\em and} the need to build a community that can share debugging
methods and infrastructure: the DOE report mentioned
earlier \cite{hpcdoe},
an NSF workshop~\cite{Cohen:2018:IRC:3297279}, and an ASCR report on
extreme heterogeneity~\cite{ascr-report-extreme-heterogeneity}.
%
Our key contribution in this paper is a fresh approach to debugging
that (1)~incorporates methods to debug across the API-stack
by resorting to binary tracing and thereby being able to ``dial into''
MPI bugs and/or OpenMP bugs (as shown in the ILCS case study), (2)~makes
an initial triage of debugging methods possible via function-call traces,
and (3)~enables the verification community to cohere around DiffTrace
by allowing other tools to extend our tool chain (they can tap into it at various places).


Many HPC debugging efforts have emphasized
the need to highlight dissimilarities and
incorporate progress measures on loops. We now
summarize a few of them.
%
AutomaDeD~\cite{automaded-GBron}\cite{automaded-laguna}
captures the application's control flow
via Semi Markov Models and detects outlier executions.
%
PRODOMETER~\cite{prodometer} detects loops in
AutomaDeD models and introduces the
notion of {\em least progressed tasks} by analyzing {\em progress dependency graphs}.
%
DiffTrace's DiffNLR method does not (yet) incorporate progress measures; it only
computes changes in loop {\em structure}.
%
Prodometer's methods are ripe for symbiotic incorporation into DiffTrace.
%
We also plan to incorporate {\em happens-before} computation as a progress measure using FCA-based algorithms by Garg et al.~\cite{latticeForDistConst,garg_2015}.
%
FCA-based approaches have been widely used in data mining~\cite{cldm},
machine learning~\cite{clml}, and information retrieval \cite{ignatov17}.


In terms of computing differences with previous executions,
we draw inspirations from
Zeller's delta-debugging~\cite{DBLP:conf/esec/Zeller99}
and De Rose et al.'s relative debugging~\cite{relative-debugging}.
%
The power of equivalence classes for outlier detection is
researched in STAT~\cite{stat}, which
merges stack traces from processes into a prefix tree,
looking for equivalence-class outliers.
%
STAT uses the StackWalker API from Dyninst~\cite{dyninst} to gather stack traces
and efficiently handles scaling issues
through tree-based overlay networks such as MRnet~\cite{mrnet}.
%
D4~\cite{liu-18} detects concurrency bugs by statically analyzing source-code
changes, and DMTracker~\cite{dmtracker} detects anomalies in data movement.
%
The communication patterns of HPC applications can be automatically characterized by
diffing the communication matrix with common patterns~\cite{roth-15} or by
detecting repetitive patterns~\cite{preissl-08}.
%
ScalaTrace~\cite{scalatrace} captures and compresses communication traces for later replay. 
%
Synoptic~\cite{beschastnikh-synoptic} is applied to distributed
system logs to find bugs.
%
Ravel~\cite{ravel} systematically visualizes
large-scale application communications~\cite{charmVis}. 

%
%
%\subsection{OTHERS}
%\hl{in case of lack of material}
%\begin{itemize}
%\item Trace File Comparison with a hierarchical Sequence Alignment algorithm \cite{weber-seqAlign}
%\item structural clustering : matthias weber \cite{weberStructural}
%\item building a better backtrace: techniques for postmortem program analysis - ben liblit \cite{liblit02}
%\item automatically charecterizing large scale program behavior - timothy sherwood \cite{sherwood02}
%\item Score-P \cite{scorep}
%\item TAU \cite{tau}
%\item ScalaTrace: Scalable compression and replay of communication traces for HPC  \cite{scalatrace}
%\end{itemize}






%
%\subsection{STAT}
%
%Parallel debugger STAT\cite{stat}
%\begin{itemize}
%\item STAT gathers stack traces from all processes
%\item Merge them into prefix tree
%\item Groups processes that exhibit similar behavior into equivalent classes
%\item A single representative of each equivalence can then be examined with a full-featured debugger like TotalView or DDT
%\end{itemize}
%
%What STAT does not have?
%
%\begin{itemize}
%\item FP debugging
%\item Portability (too many dependencies)
%\item Domain-specific
%\item Loop structures and detection
%\end{itemize}

