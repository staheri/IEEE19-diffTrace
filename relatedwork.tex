

aims to build

Similar to DiffTrace, majority of HPC debuggers analysis is based on computing  \textit{dissimilarities}, detecting \textit{anomalies} and measuring \textit{progress}.
%
AutomaDeD~\cite{automaded-GBron}\cite{automaded-laguna} statically models the application's control flow into Semi Markov Models and detects outlier states of the model in the suspicious period of time.
%
STAT~\cite{stat} merges stack traces from all processes into a prefix tree, and identifies equivalent classes of processes and outliers for further study.
%
STAT uses StackWalker API from Dyninst~\cite{dyninst} to gather stack traces, and handles the movement of traces by deploying a tree based overlay network via MRnet~\cite{mrnet}.
%
PRODOMETER~\cite{prodometer} detect loops in AutomaDeD models and introduces the least progressed tasks via analyzing a progress dependency graph.
%
D4~\cite{liu-18} detects concurrent bugs from statically analyzing source codes \textit{changes} and DMTracker~\cite{dmtracker} detects anomaly in data movements.

Obtaining and analyzing a \textit{model} from execution data is another common approach in HPC debugging and program understanding. 
Synoptic~\cite{beschastnikh-synoptic} and similar works~\cite{beschastnikh-pologs}\cite{beschastnikh-csight} derive and infer different models and invariants from logs of concurrent and distributed systems.
%
Communication patterns of HPC applications are automatically characterized by diffing the communication matrix with common known patterns in~\cite{roth-15} and by detecting repetitive patterns in \cite{preissl-08}.
%
ScalaTrace~\cite{scalatrace} also captures and compresses communication traces for later replay. 
%

FCA-based approaches have been widely used  in data mining~\cite{cldm}, machine learning~\cite{clml} and information retrieval \cite{ignatov17}. Vijay Garg uses lattic theories and algorithms to model and analyze distributed systems~\cite{latticeForDistConst}\cite{garg_2015}.

Visualizing HPC execution traces is another effort that DiffTrace and others made towards program understanding. Ravel~\cite{ravel} systematically visualizes large-scale application communication by assigning logical ideas to trace events~\cite{charmVis}. 







\subsection{OTHERS}
\hl{in case of lack of material}
\begin{itemize}
\item Trace File Comparison with a hierarchical Sequence Alignment algorithm \cite{weber-seqAlign}
\item structural clustering : matthias weber \cite{weberStructural}
\item building a better backtrace: techniques for postmortem program analysis - ben liblit \cite{liblit02}
\item automatically charecterizing large scale program behavior - timothy sherwood \cite{sherwood02}
\item Score-P \cite{scorep}
\item TAU \cite{tau}
\item ScalaTrace: Scalable compression and replay of communication traces for HPC  \cite{scalatrace}
\end{itemize}






%
%\subsection{STAT}
%
%Parallel debugger STAT\cite{stat}
%\begin{itemize}
%\item STAT gathers stack traces from all processes
%\item Merge them into prefix tree
%\item Groups processes that exhibit similar behavior into equivalent classes
%\item A single representative of each equivalence can then be examined with a full-featured debugger like TotalView or DDT
%\end{itemize}
%
%What STAT does not have?
%
%\begin{itemize}
%\item FP debugging
%\item Portability (too many dependencies)
%\item Domain-specific
%\item Loop structures and detection
%\end{itemize}
