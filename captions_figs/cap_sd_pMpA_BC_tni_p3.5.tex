\caption{The purpose of this table is to show that ParLOT is doing better job on larger input sizes. NAS benchmark input sizes are as follows : $size(A) < size(B) < size (C) < size(D) $. The overhead it adds to the application is smaller for input size C and I believe the reason is the redundant captured data (function calls) for each run helps the performance of compressing process, thus helps the overall performance. I can run these experiments with smaller input size (A) or larger (D) and include them in this table. Running NAS applications with \textit{A} makes running times so small (less than a second for some of applications) and running with \textit{D} is going to consume a lot of SUs, if we decide to do that. Also these are the results when I use the latest version of \textit{Pin} which is \textbf{3.5}. Table \ref{sd_pMpA_BC_tni_p3.0} (next one) is identical to this table but using version \textbf{3.0} of \textit{Pin}}