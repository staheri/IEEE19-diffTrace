Effective HPC debugging of many classes of bugs on heterogeneous
platforms requires efficient tracing methods that can track control
flow in terms of function calls and returns, and also optionally
captures a record of executions through user and system codes.
%
Binary instrumentation is the only approach that can accomplish this
goal; unfortunately, available tools either do not produce the kinds
of traces that can help with debugging or incur huge overheads when
the traces are brought out of the cores.
%
In this paper, we present \parlot,  a tool that extends Intel's \pin
tool to not only perform binary-level tracing, but adds several  key
features: (1)~it deploys a set of highly efficient and programmable
trace compression methods that reduces the trace volume exported 
from a chip dramatically, thus making tracing affordable.
(2)~it extends \pin's tracing methods to not only record function
calls and returns, but using heuristic approaches corrects the
stack pointer to accommodate non-standard returns.
(3)~it produces a wealth of statistics including caller/callee
relations, call frequencies, and a linear trace of entire executions
at the granularity the user opts for.
%
This paper establishes that comparable capabilities are 
unavailable by evaluating the best alternative tracing options
on runs up to 1,024 cores on the NAS parallel benchmarks on 
two HPC platforms.
%
Our experiments show that parlot can produce full function call trace with only 0.85 kB required bandwidth while slowing down applications only by as low as 38\%(fig \ref{sd.pin.cg}).
%
%- 
%- Some parallel-based bugs only manifest themselves when a program is executed at scale. 
%- Such bugs are expensive to reproduce and collecting information about the flow of execution tends to be challenging due to overhead it adds to the application in term of runtime and memory. 
%- Existing profiler and debugger tools for large applications either add too 
%- much overhead to the actual application or their collected data does not bring significant insight about the execution flow. 
%- We believe gathering enough information at low cost using a light-weight on-the-fly incremental compression mechanism would overcome large-scale verification challenges. 
%- We implemented our idea in a proof-of-concept tool called ParLOT that incrementally compresses the gathered information before it is written to memory or disk. 
%- 
%- ParLOT can track every function call entry and exit event of any HPC applications running on supercomputers with low overhead added to execution time while compressing the collected information by a factor of 100, resulting in only a few kilobytes per second of trace data being emitted by each processor.
%- 
