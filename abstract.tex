The need for efficient tracing to gain better insight from
HPC application executions is growing. 
%
An efficient and easy-to-use tracing method that overcome the trade-off
between maximum information with minimum added overhead
sounds essential for today's HPC application developers, who might
not have enough expertise in parallel programming.
% 
Unfortunately, available tools either do not produce the kinds
of traces that can help with program understanding and debugging,
or incur huge overheads when
the traces are brought out of the cores.
%
%
In this paper, we present \parlot, a tool that provides several key
features: (1)~It dynamically instruments the binary without need of source-code modification or recompilation
%
(2)~It deploys a set of highly efficient trace compression methods that reduces the trace volume gathered at runtime dramatically, thus making tracing cheaper.
%
(3)~It supports program execution analysis via tracing by
including caller/callee
relations, call frequencies, and a linear trace of entire executions
at the granularity the user opts for.
%
This paper establishes that comparable capabilities are 
unavailable by evaluating the best alternative tracing options
on runs up to 1,024 cores on the NAS parallel benchmarks on the
Comet supercomputer.
%
Our experiments show that \parlot can produce whole-program function 
call traces with average of 56 kB/s required bandwidth while 
slowing down applications by 2.7x on average. 
