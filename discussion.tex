DiffTrace is the first tool we know of that situates debugging around {\em whole program}
diffing, provides user-selectable front-end filters of function calls to keep,
summarizes loops based on state-of-the-art algorithms to detect loop-level
behavioral differences,
condenses the loop-summarized
traces into concept lattices that are built using incremental
algorithms, and clusters behaviors using hierarchical clustering and ranks them by similarity to detect and highlight the most salient differences.
%
We deliberately chose the path of a clean start that addresses missing features
in existing tools and missing collectivism in the debugging community.
%
The initial assessment of our design presented in this paper is encouraging.


Scalability, parameterization, and machine learning are our three remaining major tasks.
%
Debugging is always scalability-challenged because of growing application scales.
%
Modularity and scalability can be achieved by handing off detailed debugging from
DiffTrace to specialized tools such as Archer~\cite{archer}.
% that our group has developed and that is used in DOE lab projects.
%
One can gain scalability by further developing DiffTrace's mechanisms to filter function
calls, improving on ParLoT, and discovering semantic clustering
methods that are tuned to the bug class being worked on.
%
We plan to research the role machine learning can play by using concept lattices
and loop structures as elevated features and by conducting systematic bug-injection
studies to see whether bug classifications can be computed with high precision
and recall.
%
Our one-year plans are to take up more challenging examples, beginning with DOE miniapps, stabilize DiffTrace, and release it to the community.
 
%--end

