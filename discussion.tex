DiffTrace is the first tool we know of that situates debugging around {\em whole program}
diffing provides user-selectable front-end filters of function calls to keep,
summarizes loops based on state-of-the-art algorithms to detect loop-level
behavioral differences,
condenses the loop-summarized
traces into concept lattices that are built using incremental
algorithms, and clusters behaviors using hierarchical clustering and ranks them by similarity to detect and highlight the most salient differences.
%a
We deliberately chose the path of a clean start that addresses missing features
in existing tools and missing collectivism in the debugging community.
%
The initial assessment of our design presented in this paper is encouraging.
%

As our ongoing and future plans, we aim to stabilize, improve, and expand DiffTrace components by: 
%
%
(1)~optimizing DiffTrace components to exploit multi-core CPUs, thus reducing the overall analysis time.  Our initial results in this regard are quite promising. 
%
(2)~converting ParLOT traces into Open Trace Format (OTF2)~\cite{otf2} by logically timestamping trace entries to mine temporal properties of functions such as \textit{happened-before}~\cite{lamport}
%
(3)~conducting systematic bug-injection to see whether concept lattices and loop structures can be used as elevated features for precise bug classifications via machine learning and neural network techniques.
%
(4)~taking up more challenging and real-world examples to evaluate DiffTrace against similar tools, and release it to the community.
%--end

