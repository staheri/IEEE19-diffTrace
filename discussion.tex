DiffTrace is the first tool we know of that situates debugging around {\em whole program}
diffing, provides user-selectable front-end filters of function calls to keep,
summarizes loops based on state-of-the-art algorithms to detect loop-level
behavioral differences,
condenses the loop-summarized
traces into concept lattices that are built using incremental
algorithms, and clusters behaviors using hierarchical clustering and ranks them by similarity to detect and highlight the most salient differences.
%a
We deliberately chose the path of a clean start that addresses missing features
in existing tools and missing collectivism in the debugging community.
%
The initial assessment of our design presented in this paper is encouraging.
\hl{
After we submitted the paper, we parallelized the trace preprocessing phase of DiffTrace using 24 OpenMP threads and gained up to \textbf{31x} and on average \textbf{10x} speedup. 


Timestamping trace entries, parameterization, and machine learning are our three remaining major tasks.
%
The idea is to convert ParLOT traces into Open Trace Format (OTF2)}~\cite{otf2}\hl{, a widely used format by existing tools such as Score-p and TAU, by assigning a logical timestamp to each entry per trace.
%
Traces then can be visualized and Happens-Before}~\cite{lamport}\hl{ relation can be mined to infer about the flow of the program formally.}
%
One can gain scalability by further developing DiffTrace's mechanisms to filter function
calls, improving on ParLOT, and discovering semantic clustering
methods that are tuned to the bug class being worked on.
%
We plan to research the role machine learning can play by using concept lattices
and loop structures as elevated features and by conducting systematic bug-injection
studies to see whether bug classifications can be computed with high precision
and recall.
%
Our one-year plans are to take up more challenging examples, beginning with DOE miniapps, stabilize DiffTrace \hl{and evaluate it against similar tools}, and release it to the community.
 
%--end

